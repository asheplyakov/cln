\input texinfo  @c -*-texinfo-*-
@c %**start of header
@setfilename cln.info
@settitle CLN, a Class Library for Numbers
@c @setchapternewpage off
@c For `info' only.
@paragraphindent 0
@c For TeX only.
@iftex
@c I hate putting "@noindent" in front of every paragraph.
@parindent=0pt
@end iftex
@c %**end of header

@direntry
* CLN: (cln).                       Class Library for Numbers (C++).
@end direntry

@c My own index.
@defindex my
@c Don't need the other types of indices.
@synindex cp my
@synindex fn my
@synindex vr my
@synindex ky my
@synindex pg my
@synindex tp my


@c For `info' only.
@ifinfo
This file documents @sc{cln}, a Class Library for Numbers.

Published by Bruno Haible, @code{<haible@@clisp.cons.org>} and
Richard B. Kreckel, @code{<kreckel@@ginac.de>}.

Copyright (C)  Bruno Haible 1995, 1996, 1997, 1998, 1999, 2000, 2001, 2002, 2003, 2004, 2005, 2006.
Copyright (C)  Richard B. Kreckel 2000, 2001, 2002, 2003, 2004, 2005, 2006.

Permission is granted to make and distribute verbatim copies of
this manual provided the copyright notice and this permission notice
are preserved on all copies.

@ignore
Permission is granted to process this file through TeX and print the
results, provided the printed document carries copying permission
notice identical to this one except for the removal of this paragraph
(this paragraph not being relevant to the printed manual).

@end ignore
Permission is granted to copy and distribute modified versions of this
manual under the conditions for verbatim copying, provided that the entire
resulting derived work is distributed under the terms of a permission
notice identical to this one.

Permission is granted to copy and distribute translations of this manual
into another language, under the above conditions for modified versions,
except that this permission notice may be stated in a translation approved
by the author.
@end ifinfo


@c For TeX only.
@c prevent ugly black rectangles on overfull hbox lines:
@finalout
@titlepage
@title CLN, a Class Library for Numbers

@author by Bruno Haible
@page
@vskip 0pt plus 1filll
Copyright @copyright{} Bruno Haible 1995, 1996, 1997, 1998, 1999, 2000, 2001, 2002, 2003, 2004, 2005.
@sp 0
Copyright @copyright{} Richard Kreckel 2000, 2001, 2002, 2003, 2004, 2005.

@sp 2
Published by Bruno Haible, @code{<haible@@clisp.cons.org>} and
Richard Kreckel, @code{<kreckel@@ginac.de>}.

Permission is granted to make and distribute verbatim copies of
this manual provided the copyright notice and this permission notice
are preserved on all copies.

Permission is granted to copy and distribute modified versions of this
manual under the conditions for verbatim copying, provided that the entire
resulting derived work is distributed under the terms of a permission
notice identical to this one.

Permission is granted to copy and distribute translations of this manual
into another language, under the above conditions for modified versions,
except that this permission notice may be stated in a translation approved
by the author.

@end titlepage
@page


@c Table of contents
@contents


@node Top, Introduction, (dir), (dir)

@c @menu
@c * Introduction::                Introduction
@c @end menu


@node Introduction, Top, Top, Top
@comment node-name, next, previous, up
@chapter Introduction

@noindent
CLN is a library for computations with all kinds of numbers.
It has a rich set of number classes:

@itemize @bullet
@item
Integers (with unlimited precision),

@item
Rational numbers,

@item
Floating-point numbers:

@itemize @minus
@item
Short float,
@item
Single float,
@item
Double float,
@item
Long float (with unlimited precision),
@end itemize

@item
Complex numbers,

@item
Modular integers (integers modulo a fixed integer),

@item
Univariate polynomials.
@end itemize

@noindent
The subtypes of the complex numbers among these are exactly the
types of numbers known to the Common Lisp language. Therefore
@code{CLN} can be used for Common Lisp implementations, giving
@samp{CLN} another meaning: it becomes an abbreviation of
``Common Lisp Numbers''.

@noindent
The CLN package implements

@itemize @bullet
@item
Elementary functions (@code{+}, @code{-}, @code{*}, @code{/}, @code{sqrt},
comparisons, @dots{}),

@item
Logical functions (logical @code{and}, @code{or}, @code{not}, @dots{}),

@item
Transcendental functions (exponential, logarithmic, trigonometric, hyperbolic
functions and their inverse functions).
@end itemize

@noindent
CLN is a C++ library. Using C++ as an implementation language provides

@itemize @bullet
@item
efficiency: it compiles to machine code,
@item
type safety: the C++ compiler knows about the number types and complains
if, for example, you try to assign a float to an integer variable.
@item
algebraic syntax: You can use the @code{+}, @code{-}, @code{*}, @code{=},
@code{==}, @dots{} operators as in C or C++.
@end itemize

@noindent
CLN is memory efficient:

@itemize @bullet
@item
Small integers and short floats are immediate, not heap allocated.
@item
Heap-allocated memory is reclaimed through an automatic, non-interruptive
garbage collection.
@end itemize

@noindent
CLN is speed efficient:

@itemize @bullet
@item
The kernel of CLN has been written in assembly language for some CPUs
(@code{i386}, @code{m68k}, @code{sparc}, @code{mips}, @code{arm}).
@item
@cindex GMP
On all CPUs, CLN may be configured to use the superefficient low-level
routines from GNU GMP version 3.
@item
It uses Karatsuba multiplication, which is significantly faster
for large numbers than the standard multiplication algorithm.
@item
For very large numbers (more than 12000 decimal digits), it uses
@iftex
Sch{@"o}nhage-Strassen
@cindex Sch{@"o}nhage-Strassen multiplication
@end iftex
@ifinfo
Schnhage-Strassen
@cindex Schnhage-Strassen multiplication
@end ifinfo
multiplication, which is an asymptotically optimal multiplication
algorithm, for multiplication, division and radix conversion.
@end itemize

@noindent
CLN aims at being easily integrated into larger software packages:

@itemize @bullet
@item
The garbage collection imposes no burden on the main application.
@item
The library provides hooks for memory allocation and exceptions.
@item
@cindex namespace
All non-macro identifiers are hidden in namespace @code{cln} in 
order to avoid name clashes.
@end itemize


@chapter Installation

This section describes how to install the CLN package on your system.


@section Prerequisites

@subsection C++ compiler

To build CLN, you need a C++ compiler.
Actually, you need GNU @code{g++ 2.95} or newer.

The following C++ features are used:
classes, member functions, overloading of functions and operators,
constructors and destructors, inline, const, multiple inheritance,
templates and namespaces.

The following C++ features are not used:
@code{new}, @code{delete}, virtual inheritance, exceptions.

CLN relies on semi-automatic ordering of initializations of static and
global variables, a feature which I could implement for GNU g++
only. Also, it is not known whether this semi-automatic ordering works
on all platforms when a non-GNU assembler is being used.

@subsection Make utility
@cindex @code{make}

To build CLN, you also need to have GNU @code{make} installed.

Only GNU @code{make} 3.77 is unusable for CLN; other versions work fine.

@subsection Sed utility
@cindex @code{sed}

To build CLN on HP-UX, you also need to have GNU @code{sed} installed.
This is because the libtool script, which creates the CLN library, relies
on @code{sed}, and the vendor's @code{sed} utility on these systems is too
limited.


@section Building the library

As with any autoconfiguring GNU software, installation is as easy as this:

@example
$ ./configure
$ make
$ make check
@end example

If on your system, @samp{make} is not GNU @code{make}, you have to use
@samp{gmake} instead of @samp{make} above.

The @code{configure} command checks out some features of your system and
C++ compiler and builds the @code{Makefile}s. The @code{make} command
builds the library. This step may take about an hour on an average workstation.
The @code{make check} runs some test to check that no important subroutine
has been miscompiled.

The @code{configure} command accepts options. To get a summary of them, try

@example
$ ./configure --help
@end example

Some of the options are explained in detail in the @samp{INSTALL.generic} file.

You can specify the C compiler, the C++ compiler and their options through
the following environment variables when running @code{configure}:

@table @code
@item CC
Specifies the C compiler.

@item CFLAGS
Flags to be given to the C compiler when compiling programs (not when linking).

@item CXX
Specifies the C++ compiler.

@item CXXFLAGS
Flags to be given to the C++ compiler when compiling programs (not when linking).
@end table

Examples:

@example
$ CC="gcc" CFLAGS="-O" CXX="g++" CXXFLAGS="-O" ./configure
$ CC="gcc -V egcs-2.91.60" CFLAGS="-O -g" \
  CXX="g++ -V egcs-2.91.60" CXXFLAGS="-O -g" ./configure
$ CC="gcc -V 2.95.2" CFLAGS="-O2 -fno-exceptions" \
  CXX="g++ -V 2.95.2" CFLAGS="-O2 -fno-exceptions" ./configure
$ CC="gcc -V 3.0.4" CFLAGS="-O2 -finline-limit=1000 -fno-exceptions" \
  CXX="g++ -V 3.0.4" CFLAGS="-O2 -finline-limit=1000 -fno-exceptions" \
  ./configure
@end example

Note that for these environment variables to take effect, you have to set
them (assuming a Bourne-compatible shell) on the same line as the
@code{configure} command. If you made the settings in earlier shell
commands, you have to @code{export} the environment variables before
calling @code{configure}. In a @code{csh} shell, you have to use the
@samp{setenv} command for setting each of the environment variables.

Currently CLN works only with the GNU @code{g++} compiler, and only in
optimizing mode. So you should specify at least @code{-O} in the CXXFLAGS,
or no CXXFLAGS at all. (If CXXFLAGS is not set, CLN will use @code{-O}.)

If you use @code{g++} 3.x, I recommend adding @samp{-finline-limit=1000}
to the CXXFLAGS. This is essential for good code.

If you use @code{g++} gcc-2.95.x or gcc-3.x , I recommend adding
@samp{-fno-exceptions} to the CXXFLAGS. This will likely generate better code.

If you use @code{g++} from gcc-3.0.4 or older on Sparc, add either
@samp{-O}, @samp{-O1} or @samp{-O2 -fno-schedule-insns} to the
CXXFLAGS. With full @samp{-O2}, @code{g++} miscompiles the division
routines. If you use @code{g++} older than 2.95.3 on Sparc you should
also specify @samp{--disable-shared} because of bad code produced in the
shared library. Also, do not use gcc-3.0 on Sparc for compiling CLN, it
won't work at all.

If you use @code{g++} on OSF/1 or Tru64 using gcc-2.95.x, you should
specify @samp{--disable-shared} because of linker problems with
duplicate symbols in shared libraries.  If you use @code{g++} from
gcc-3.0.n, with n larger than 1, you should @emph{not} add
@samp{-fno-exceptions} to the CXXFLAGS, since that will generate wrong
code (gcc-3.1 is okay again, as is gcc-3.0).

Also, please do not compile CLN with @code{g++} using the @code{-O3}
optimization level.  This leads to inferior code quality.

If you use @code{g++} from gcc-3.1, it will need 235 MB of virtual memory.
You might need some swap space if your machine doesn't have 512 MB of RAM.

By default, both a shared and a static library are built.  You can build
CLN as a static (or shared) library only, by calling @code{configure} with
the option @samp{--disable-shared} (or @samp{--disable-static}).  While
shared libraries are usually more convenient to use, they may not work
on all architectures.  Try disabling them if you run into linker
problems.  Also, they are generally somewhat slower than static
libraries so runtime-critical applications should be linked statically.

If you use @code{g++} from gcc-3.1 with option @samp{-g}, you will need
some disk space: 335 MB for building as both a shared and a static library,
or 130 MB when building as a shared library only.


@subsection Using the GNU MP Library
@cindex GMP

Starting with version 1.1, CLN may be configured to make use of a
preinstalled @code{gmp} library.  Please make sure that you have at
least @code{gmp} version 3.0 installed since earlier versions are
unsupported and likely not to work.  Enabling this feature by calling
@code{configure} with the option @samp{--with-gmp} is known to be quite
a boost for CLN's performance.

If you have installed the @code{gmp} library and its header file in
some place where your compiler cannot find it by default, you must help
@code{configure} by setting @code{CPPFLAGS} and @code{LDFLAGS}.  Here is
an example:

@example
$ CC="gcc" CFLAGS="-O2" CXX="g++" CXXFLAGS="-O2 -fno-exceptions" \
  CPPFLAGS="-I/opt/gmp/include" LDFLAGS="-L/opt/gmp/lib" ./configure --with-gmp
@end example


@section Installing the library
@cindex installation

As with any autoconfiguring GNU software, installation is as easy as this:

@example
$ make install
@end example

The @samp{make install} command installs the library and the include files
into public places (@file{/usr/local/lib/} and @file{/usr/local/include/},
if you haven't specified a @code{--prefix} option to @code{configure}).
This step may require superuser privileges.

If you have already built the library and wish to install it, but didn't
specify @code{--prefix=@dots{}} at configure time, just re-run
@code{configure}, giving it the same options as the first time, plus
the @code{--prefix=@dots{}} option.


@section Cleaning up

You can remove system-dependent files generated by @code{make} through

@example
$ make clean
@end example

You can remove all files generated by @code{make}, thus reverting to a
virgin distribution of CLN, through

@example
$ make distclean
@end example


@chapter Ordinary number types

CLN implements the following class hierarchy:

@example
                        Number
                      cl_number
                    <cln/number.h>
                          |
                          |
                 Real or complex number
                        cl_N
                    <cln/complex.h>
                          |
                          |
                     Real number
                        cl_R
                     <cln/real.h>
                          |
      +-------------------+-------------------+
      |                                       |
Rational number                     Floating-point number
    cl_RA                                   cl_F
<cln/rational.h>                         <cln/float.h>
      |                                       |
      |                +--------------+--------------+--------------+
   Integer             |              |              |              |
    cl_I          Short-Float    Single-Float   Double-Float    Long-Float
<cln/integer.h>      cl_SF          cl_FF          cl_DF          cl_LF
                 <cln/sfloat.h> <cln/ffloat.h> <cln/dfloat.h> <cln/lfloat.h>
@end example

@cindex @code{cl_number}
@cindex abstract class
The base class @code{cl_number} is an abstract base class.
It is not useful to declare a variable of this type except if you want
to completely disable compile-time type checking and use run-time type
checking instead.

@cindex @code{cl_N}
@cindex real number
@cindex complex number
The class @code{cl_N} comprises real and complex numbers. There is
no special class for complex numbers since complex numbers with imaginary
part @code{0} are automatically converted to real numbers.

@cindex @code{cl_R}
The class @code{cl_R} comprises real numbers of different kinds. It is an
abstract class.

@cindex @code{cl_RA}
@cindex rational number
@cindex integer
The class @code{cl_RA} comprises exact real numbers: rational numbers, including
integers. There is no special class for non-integral rational numbers
since rational numbers with denominator @code{1} are automatically converted
to integers.

@cindex @code{cl_F}
The class @code{cl_F} implements floating-point approximations to real numbers.
It is an abstract class.


@section Exact numbers
@cindex exact number

Some numbers are represented as exact numbers: there is no loss of information
when such a number is converted from its mathematical value to its internal
representation. On exact numbers, the elementary operations (@code{+},
@code{-}, @code{*}, @code{/}, comparisons, @dots{}) compute the completely
correct result.

In CLN, the exact numbers are:

@itemize @bullet
@item
rational numbers (including integers),
@item
complex numbers whose real and imaginary parts are both rational numbers.
@end itemize

Rational numbers are always normalized to the form
@code{@var{numerator}/@var{denominator}} where the numerator and denominator
are coprime integers and the denominator is positive. If the resulting
denominator is @code{1}, the rational number is converted to an integer.

@cindex immediate numbers
Small integers (typically in the range @code{-2^29}@dots{}@code{2^29-1},
for 32-bit machines) are especially efficient, because they consume no heap
allocation. Otherwise the distinction between these immediate integers
(called ``fixnums'') and heap allocated integers (called ``bignums'')
is completely transparent.


@section Floating-point numbers
@cindex floating-point number

Not all real numbers can be represented exactly. (There is an easy mathematical
proof for this: Only a countable set of numbers can be stored exactly in
a computer, even if one assumes that it has unlimited storage. But there
are uncountably many real numbers.) So some approximation is needed.
CLN implements ordinary floating-point numbers, with mantissa and exponent.

@cindex rounding error
The elementary operations (@code{+}, @code{-}, @code{*}, @code{/}, @dots{})
only return approximate results. For example, the value of the expression
@code{(cl_F) 0.3 + (cl_F) 0.4} prints as @samp{0.70000005}, not as
@samp{0.7}. Rounding errors like this one are inevitable when computing
with floating-point numbers.

Nevertheless, CLN rounds the floating-point results of the operations @code{+},
@code{-}, @code{*}, @code{/}, @code{sqrt} according to the ``round-to-even''
rule: It first computes the exact mathematical result and then returns the
floating-point number which is nearest to this. If two floating-point numbers
are equally distant from the ideal result, the one with a @code{0} in its least
significant mantissa bit is chosen.

Similarly, testing floating point numbers for equality @samp{x == y}
is gambling with random errors. Better check for @samp{abs(x - y) < epsilon}
for some well-chosen @code{epsilon}.

Floating point numbers come in four flavors:

@itemize @bullet
@item
@cindex @code{cl_SF}
Short floats, type @code{cl_SF}.
They have 1 sign bit, 8 exponent bits (including the exponent's sign),
and 17 mantissa bits (including the ``hidden'' bit).
They don't consume heap allocation.

@item
@cindex @code{cl_FF}
Single floats, type @code{cl_FF}.
They have 1 sign bit, 8 exponent bits (including the exponent's sign),
and 24 mantissa bits (including the ``hidden'' bit).
In CLN, they are represented as IEEE single-precision floating point numbers.
This corresponds closely to the C/C++ type @samp{float}.

@item
@cindex @code{cl_DF}
Double floats, type @code{cl_DF}.
They have 1 sign bit, 11 exponent bits (including the exponent's sign),
and 53 mantissa bits (including the ``hidden'' bit).
In CLN, they are represented as IEEE double-precision floating point numbers.
This corresponds closely to the C/C++ type @samp{double}.

@item
@cindex @code{cl_LF}
Long floats, type @code{cl_LF}.
They have 1 sign bit, 32 exponent bits (including the exponent's sign),
and n mantissa bits (including the ``hidden'' bit), where n >= 64.
The precision of a long float is unlimited, but once created, a long float
has a fixed precision. (No ``lazy recomputation''.)
@end itemize

Of course, computations with long floats are more expensive than those
with smaller floating-point formats.

CLN does not implement features like NaNs, denormalized numbers and
gradual underflow. If the exponent range of some floating-point type
is too limited for your application, choose another floating-point type
with larger exponent range.

@cindex @code{cl_F}
As a user of CLN, you can forget about the differences between the
four floating-point types and just declare all your floating-point
variables as being of type @code{cl_F}. This has the advantage that
when you change the precision of some computation (say, from @code{cl_DF}
to @code{cl_LF}), you don't have to change the code, only the precision
of the initial values. Also, many transcendental functions have been
declared as returning a @code{cl_F} when the argument is a @code{cl_F},
but such declarations are missing for the types @code{cl_SF}, @code{cl_FF},
@code{cl_DF}, @code{cl_LF}. (Such declarations would be wrong if
the floating point contagion rule happened to change in the future.)


@section Complex numbers
@cindex complex number

Complex numbers, as implemented by the class @code{cl_N}, have a real
part and an imaginary part, both real numbers. A complex number whose
imaginary part is the exact number @code{0} is automatically converted
to a real number.

Complex numbers can arise from real numbers alone, for example
through application of @code{sqrt} or transcendental functions.


@section Conversions
@cindex conversion

Conversions from any class to any its superclasses (``base classes'' in
C++ terminology) is done automatically.

Conversions from the C built-in types @samp{long} and @samp{unsigned long}
are provided for the classes @code{cl_I}, @code{cl_RA}, @code{cl_R},
@code{cl_N} and @code{cl_number}.

Conversions from the C built-in types @samp{int} and @samp{unsigned int}
are provided for the classes @code{cl_I}, @code{cl_RA}, @code{cl_R},
@code{cl_N} and @code{cl_number}. However, these conversions emphasize
efficiency. Their range is therefore limited:

@itemize @minus
@item
The conversion from @samp{int} works only if the argument is < 2^29 and > -2^29.
@item
The conversion from @samp{unsigned int} works only if the argument is < 2^29.
@end itemize

In a declaration like @samp{cl_I x = 10;} the C++ compiler is able to
do the conversion of @code{10} from @samp{int} to @samp{cl_I} at compile time
already. On the other hand, code like @samp{cl_I x = 1000000000;} is
in error.
So, if you want to be sure that an @samp{int} whose magnitude is not guaranteed
to be < 2^29 is correctly converted to a @samp{cl_I}, first convert it to a
@samp{long}. Similarly, if a large @samp{unsigned int} is to be converted to a
@samp{cl_I}, first convert it to an @samp{unsigned long}.

Conversions from the C built-in type @samp{float} are provided for the classes
@code{cl_FF}, @code{cl_F}, @code{cl_R}, @code{cl_N} and @code{cl_number}.

Conversions from the C built-in type @samp{double} are provided for the classes
@code{cl_DF}, @code{cl_F}, @code{cl_R}, @code{cl_N} and @code{cl_number}.

Conversions from @samp{const char *} are provided for the classes
@code{cl_I}, @code{cl_RA},
@code{cl_SF}, @code{cl_FF}, @code{cl_DF}, @code{cl_LF}, @code{cl_F},
@code{cl_R}, @code{cl_N}.
The easiest way to specify a value which is outside of the range of the
C++ built-in types is therefore to specify it as a string, like this:
@cindex Rubik's cube
@example
   cl_I order_of_rubiks_cube_group = "43252003274489856000";
@end example
Note that this conversion is done at runtime, not at compile-time.

Conversions from @code{cl_I} to the C built-in types @samp{int},
@samp{unsigned int}, @samp{long}, @samp{unsigned long} are provided through
the functions

@table @code
@item int cl_I_to_int (const cl_I& x)
@cindex @code{cl_I_to_int ()}
@itemx unsigned int cl_I_to_uint (const cl_I& x)
@cindex @code{cl_I_to_uint ()}
@itemx long cl_I_to_long (const cl_I& x)
@cindex @code{cl_I_to_long ()}
@itemx unsigned long cl_I_to_ulong (const cl_I& x)
@cindex @code{cl_I_to_ulong ()}
Returns @code{x} as element of the C type @var{ctype}. If @code{x} is not
representable in the range of @var{ctype}, a runtime error occurs.
@end table

Conversions from the classes @code{cl_I}, @code{cl_RA},
@code{cl_SF}, @code{cl_FF}, @code{cl_DF}, @code{cl_LF}, @code{cl_F} and
@code{cl_R}
to the C built-in types @samp{float} and @samp{double} are provided through
the functions

@table @code
@item float float_approx (const @var{type}& x)
@cindex @code{float_approx ()}
@itemx double double_approx (const @var{type}& x)
@cindex @code{double_approx ()}
Returns an approximation of @code{x} of C type @var{ctype}.
If @code{abs(x)} is too close to 0 (underflow), 0 is returned.
If @code{abs(x)} is too large (overflow), an IEEE infinity is returned.
@end table

Conversions from any class to any of its subclasses (``derived classes'' in
C++ terminology) are not provided. Instead, you can assert and check
that a value belongs to a certain subclass, and return it as element of that
class, using the @samp{As} and @samp{The} macros.
@cindex cast
@cindex @code{As()()}
@code{As(@var{type})(@var{value})} checks that @var{value} belongs to
@var{type} and returns it as such.
@cindex @code{The()()}
@code{The(@var{type})(@var{value})} assumes that @var{value} belongs to
@var{type} and returns it as such. It is your responsibility to ensure
that this assumption is valid.  Since macros and namespaces don't go
together well, there is an equivalent to @samp{The}: the template
@samp{the}.

Example:

@example
@group
   cl_I x = @dots{};
   if (!(x >= 0)) abort();
   cl_I ten_x_a = The(cl_I)(expt(10,x)); // If x >= 0, 10^x is an integer.
                // In general, it would be a rational number.
   cl_I ten_x_b = the<cl_I>(expt(10,x)); // The same as above.
@end group
@end example


@chapter Functions on numbers

Each of the number classes declares its mathematical operations in the
corresponding include file. For example, if your code operates with
objects of type @code{cl_I}, it should @code{#include <cln/integer.h>}.


@section Constructing numbers

Here is how to create number objects ``from nothing''.


@subsection Constructing integers

@code{cl_I} objects are most easily constructed from C integers and from
strings. See @ref{Conversions}.


@subsection Constructing rational numbers

@code{cl_RA} objects can be constructed from strings. The syntax
for rational numbers is described in @ref{Internal and printed representation}.
Another standard way to produce a rational number is through application
of @samp{operator /} or @samp{recip} on integers.


@subsection Constructing floating-point numbers

@code{cl_F} objects with low precision are most easily constructed from
C @samp{float} and @samp{double}. See @ref{Conversions}.

To construct a @code{cl_F} with high precision, you can use the conversion
from @samp{const char *}, but you have to specify the desired precision
within the string. (See @ref{Internal and printed representation}.)
Example:
@example
   cl_F e = "0.271828182845904523536028747135266249775724709369996e+1_40";
@end example
will set @samp{e} to the given value, with a precision of 40 decimal digits.

The programmatic way to construct a @code{cl_F} with high precision is
through the @code{cl_float} conversion function, see
@ref{Conversion to floating-point numbers}. For example, to compute
@code{e} to 40 decimal places, first construct 1.0 to 40 decimal places
and then apply the exponential function:
@example
   float_format_t precision = float_format(40);
   cl_F e = exp(cl_float(1,precision));
@end example


@subsection Constructing complex numbers

Non-real @code{cl_N} objects are normally constructed through the function
@example
   cl_N complex (const cl_R& realpart, const cl_R& imagpart)
@end example
See @ref{Elementary complex functions}.


@section Elementary functions

Each of the classes @code{cl_N}, @code{cl_R}, @code{cl_RA}, @code{cl_I},
@code{cl_F}, @code{cl_SF}, @code{cl_FF}, @code{cl_DF}, @code{cl_LF}
defines the following operations:

@table @code
@item @var{type} operator + (const @var{type}&, const @var{type}&)
@cindex @code{operator + ()}
Addition.

@item @var{type} operator - (const @var{type}&, const @var{type}&)
@cindex @code{operator - ()}
Subtraction.

@item @var{type} operator - (const @var{type}&)
Returns the negative of the argument.

@item @var{type} plus1 (const @var{type}& x)
@cindex @code{plus1 ()}
Returns @code{x + 1}.

@item @var{type} minus1 (const @var{type}& x)
@cindex @code{minus1 ()}
Returns @code{x - 1}.

@item @var{type} operator * (const @var{type}&, const @var{type}&)
@cindex @code{operator * ()}
Multiplication.

@item @var{type} square (const @var{type}& x)
@cindex @code{square ()}
Returns @code{x * x}.
@end table

Each of the classes @code{cl_N}, @code{cl_R}, @code{cl_RA},
@code{cl_F}, @code{cl_SF}, @code{cl_FF}, @code{cl_DF}, @code{cl_LF}
defines the following operations:

@table @code
@item @var{type} operator / (const @var{type}&, const @var{type}&)
@cindex @code{operator / ()}
Division.

@item @var{type} recip (const @var{type}&)
@cindex @code{recip ()}
Returns the reciprocal of the argument.
@end table

The class @code{cl_I} doesn't define a @samp{/} operation because
in the C/C++ language this operator, applied to integral types,
denotes the @samp{floor} or @samp{truncate} operation (which one of these,
is implementation dependent). (@xref{Rounding functions}.)
Instead, @code{cl_I} defines an ``exact quotient'' function:

@table @code
@item cl_I exquo (const cl_I& x, const cl_I& y)
@cindex @code{exquo ()}
Checks that @code{y} divides @code{x}, and returns the quotient @code{x}/@code{y}.
@end table

The following exponentiation functions are defined:

@table @code
@item cl_I expt_pos (const cl_I& x, const cl_I& y)
@cindex @code{expt_pos ()}
@itemx cl_RA expt_pos (const cl_RA& x, const cl_I& y)
@code{y} must be > 0. Returns @code{x^y}.

@item cl_RA expt (const cl_RA& x, const cl_I& y)
@cindex @code{expt ()}
@itemx cl_R expt (const cl_R& x, const cl_I& y)
@itemx cl_N expt (const cl_N& x, const cl_I& y)
Returns @code{x^y}.
@end table

Each of the classes @code{cl_R}, @code{cl_RA}, @code{cl_I},
@code{cl_F}, @code{cl_SF}, @code{cl_FF}, @code{cl_DF}, @code{cl_LF}
defines the following operation:

@table @code
@item @var{type} abs (const @var{type}& x)
@cindex @code{abs ()}
Returns the absolute value of @code{x}.
This is @code{x} if @code{x >= 0}, and @code{-x} if @code{x <= 0}.
@end table

The class @code{cl_N} implements this as follows:

@table @code
@item cl_R abs (const cl_N x)
Returns the absolute value of @code{x}.
@end table

Each of the classes @code{cl_N}, @code{cl_R}, @code{cl_RA}, @code{cl_I},
@code{cl_F}, @code{cl_SF}, @code{cl_FF}, @code{cl_DF}, @code{cl_LF}
defines the following operation:

@table @code
@item @var{type} signum (const @var{type}& x)
@cindex @code{signum ()}
Returns the sign of @code{x}, in the same number format as @code{x}.
This is defined as @code{x / abs(x)} if @code{x} is non-zero, and
@code{x} if @code{x} is zero. If @code{x} is real, the value is either
0 or 1 or -1.
@end table


@section Elementary rational functions

Each of the classes @code{cl_RA}, @code{cl_I} defines the following operations:

@table @code
@item cl_I numerator (const @var{type}& x)
@cindex @code{numerator ()}
Returns the numerator of @code{x}.

@item cl_I denominator (const @var{type}& x)
@cindex @code{denominator ()}
Returns the denominator of @code{x}.
@end table

The numerator and denominator of a rational number are normalized in such
a way that they have no factor in common and the denominator is positive.


@section Elementary complex functions

The class @code{cl_N} defines the following operation:

@table @code
@item cl_N complex (const cl_R& a, const cl_R& b)
@cindex @code{complex ()}
Returns the complex number @code{a+bi}, that is, the complex number with
real part @code{a} and imaginary part @code{b}.
@end table

Each of the classes @code{cl_N}, @code{cl_R} defines the following operations:

@table @code
@item cl_R realpart (const @var{type}& x)
@cindex @code{realpart ()}
Returns the real part of @code{x}.

@item cl_R imagpart (const @var{type}& x)
@cindex @code{imagpart ()}
Returns the imaginary part of @code{x}.

@item @var{type} conjugate (const @var{type}& x)
@cindex @code{conjugate ()}
Returns the complex conjugate of @code{x}.
@end table

We have the relations

@itemize @asis
@item
@code{x = complex(realpart(x), imagpart(x))}
@item
@code{conjugate(x) = complex(realpart(x), -imagpart(x))}
@end itemize


@section Comparisons
@cindex comparison

Each of the classes @code{cl_N}, @code{cl_R}, @code{cl_RA}, @code{cl_I},
@code{cl_F}, @code{cl_SF}, @code{cl_FF}, @code{cl_DF}, @code{cl_LF}
defines the following operations:

@table @code
@item bool operator == (const @var{type}&, const @var{type}&)
@cindex @code{operator == ()}
@itemx bool operator != (const @var{type}&, const @var{type}&)
@cindex @code{operator != ()}
Comparison, as in C and C++.

@item uint32 equal_hashcode (const @var{type}&)
@cindex @code{equal_hashcode ()}
Returns a 32-bit hash code that is the same for any two numbers which are
the same according to @code{==}. This hash code depends on the number's value,
not its type or precision.

@item cl_boolean zerop (const @var{type}& x)
@cindex @code{zerop ()}
Compare against zero: @code{x == 0}
@end table

Each of the classes @code{cl_R}, @code{cl_RA}, @code{cl_I},
@code{cl_F}, @code{cl_SF}, @code{cl_FF}, @code{cl_DF}, @code{cl_LF}
defines the following operations:

@table @code
@item cl_signean compare (const @var{type}& x, const @var{type}& y)
@cindex @code{compare ()}
Compares @code{x} and @code{y}. Returns +1 if @code{x}>@code{y},
-1 if @code{x}<@code{y}, 0 if @code{x}=@code{y}.

@item bool operator <= (const @var{type}&, const @var{type}&)
@cindex @code{operator <= ()}
@itemx bool operator < (const @var{type}&, const @var{type}&)
@cindex @code{operator < ()}
@itemx bool operator >= (const @var{type}&, const @var{type}&)
@cindex @code{operator >= ()}
@itemx bool operator > (const @var{type}&, const @var{type}&)
@cindex @code{operator > ()}
Comparison, as in C and C++.

@item cl_boolean minusp (const @var{type}& x)
@cindex @code{minusp ()}
Compare against zero: @code{x < 0}

@item cl_boolean plusp (const @var{type}& x)
@cindex @code{plusp ()}
Compare against zero: @code{x > 0}

@item @var{type} max (const @var{type}& x, const @var{type}& y)
@cindex @code{max ()}
Return the maximum of @code{x} and @code{y}.

@item @var{type} min (const @var{type}& x, const @var{type}& y)
@cindex @code{min ()}
Return the minimum of @code{x} and @code{y}.
@end table

When a floating point number and a rational number are compared, the float
is first converted to a rational number using the function @code{rational}.
Since a floating point number actually represents an interval of real numbers,
the result might be surprising.
For example, @code{(cl_F)(cl_R)"1/3" == (cl_R)"1/3"} returns false because
there is no floating point number whose value is exactly @code{1/3}.


@section Rounding functions
@cindex rounding

When a real number is to be converted to an integer, there is no ``best''
rounding. The desired rounding function depends on the application.
The Common Lisp and ISO Lisp standards offer four rounding functions:

@table @code
@item floor(x)
This is the largest integer <=@code{x}.

@item ceiling(x)
This is the smallest integer >=@code{x}.

@item truncate(x)
Among the integers between 0 and @code{x} (inclusive) the one nearest to @code{x}.

@item round(x)
The integer nearest to @code{x}. If @code{x} is exactly halfway between two
integers, choose the even one.
@end table

These functions have different advantages:

@code{floor} and @code{ceiling} are translation invariant:
@code{floor(x+n) = floor(x) + n} and @code{ceiling(x+n) = ceiling(x) + n}
for every @code{x} and every integer @code{n}.

On the other hand, @code{truncate} and @code{round} are symmetric:
@code{truncate(-x) = -truncate(x)} and @code{round(-x) = -round(x)},
and furthermore @code{round} is unbiased: on the ``average'', it rounds
down exactly as often as it rounds up.

The functions are related like this:

@itemize @asis
@item
@code{ceiling(m/n) = floor((m+n-1)/n) = floor((m-1)/n)+1}
for rational numbers @code{m/n} (@code{m}, @code{n} integers, @code{n}>0), and
@item
@code{truncate(x) = sign(x) * floor(abs(x))}
@end itemize

Each of the classes @code{cl_R}, @code{cl_RA},
@code{cl_F}, @code{cl_SF}, @code{cl_FF}, @code{cl_DF}, @code{cl_LF}
defines the following operations:

@table @code
@item cl_I floor1 (const @var{type}& x)
@cindex @code{floor1 ()}
Returns @code{floor(x)}.
@item cl_I ceiling1 (const @var{type}& x)
@cindex @code{ceiling1 ()}
Returns @code{ceiling(x)}.
@item cl_I truncate1 (const @var{type}& x)
@cindex @code{truncate1 ()}
Returns @code{truncate(x)}.
@item cl_I round1 (const @var{type}& x)
@cindex @code{round1 ()}
Returns @code{round(x)}.
@end table

Each of the classes @code{cl_R}, @code{cl_RA}, @code{cl_I},
@code{cl_F}, @code{cl_SF}, @code{cl_FF}, @code{cl_DF}, @code{cl_LF}
defines the following operations:

@table @code
@item cl_I floor1 (const @var{type}& x, const @var{type}& y)
Returns @code{floor(x/y)}.
@item cl_I ceiling1 (const @var{type}& x, const @var{type}& y)
Returns @code{ceiling(x/y)}.
@item cl_I truncate1 (const @var{type}& x, const @var{type}& y)
Returns @code{truncate(x/y)}.
@item cl_I round1 (const @var{type}& x, const @var{type}& y)
Returns @code{round(x/y)}.
@end table

These functions are called @samp{floor1}, @dots{} here instead of
@samp{floor}, @dots{}, because on some systems, system dependent include
files define @samp{floor} and @samp{ceiling} as macros.

In many cases, one needs both the quotient and the remainder of a division.
It is more efficient to compute both at the same time than to perform
two divisions, one for quotient and the next one for the remainder.
The following functions therefore return a structure containing both
the quotient and the remainder. The suffix @samp{2} indicates the number
of ``return values''. The remainder is defined as follows:

@itemize @bullet
@item
for the computation of @code{quotient = floor(x)},
@code{remainder = x - quotient},
@item
for the computation of @code{quotient = floor(x,y)},
@code{remainder = x - quotient*y},
@end itemize

and similarly for the other three operations.

Each of the classes @code{cl_R}, @code{cl_RA},
@code{cl_F}, @code{cl_SF}, @code{cl_FF}, @code{cl_DF}, @code{cl_LF}
defines the following operations:

@table @code
@item struct @var{type}_div_t @{ cl_I quotient; @var{type} remainder; @};
@itemx @var{type}_div_t floor2 (const @var{type}& x)
@itemx @var{type}_div_t ceiling2 (const @var{type}& x)
@itemx @var{type}_div_t truncate2 (const @var{type}& x)
@itemx @var{type}_div_t round2 (const @var{type}& x)
@end table

Each of the classes @code{cl_R}, @code{cl_RA}, @code{cl_I},
@code{cl_F}, @code{cl_SF}, @code{cl_FF}, @code{cl_DF}, @code{cl_LF}
defines the following operations:

@table @code
@item struct @var{type}_div_t @{ cl_I quotient; @var{type} remainder; @};
@itemx @var{type}_div_t floor2 (const @var{type}& x, const @var{type}& y)
@cindex @code{floor2 ()}
@itemx @var{type}_div_t ceiling2 (const @var{type}& x, const @var{type}& y)
@cindex @code{ceiling2 ()}
@itemx @var{type}_div_t truncate2 (const @var{type}& x, const @var{type}& y)
@cindex @code{truncate2 ()}
@itemx @var{type}_div_t round2 (const @var{type}& x, const @var{type}& y)
@cindex @code{round2 ()}
@end table

Sometimes, one wants the quotient as a floating-point number (of the
same format as the argument, if the argument is a float) instead of as
an integer. The prefix @samp{f} indicates this.

Each of the classes
@code{cl_F}, @code{cl_SF}, @code{cl_FF}, @code{cl_DF}, @code{cl_LF}
defines the following operations:

@table @code
@item @var{type} ffloor (const @var{type}& x)
@cindex @code{ffloor ()}
@itemx @var{type} fceiling (const @var{type}& x)
@cindex @code{fceiling ()}
@itemx @var{type} ftruncate (const @var{type}& x)
@cindex @code{ftruncate ()}
@itemx @var{type} fround (const @var{type}& x)
@cindex @code{fround ()}
@end table

and similarly for class @code{cl_R}, but with return type @code{cl_F}.

The class @code{cl_R} defines the following operations:

@table @code
@item cl_F ffloor (const @var{type}& x, const @var{type}& y)
@itemx cl_F fceiling (const @var{type}& x, const @var{type}& y)
@itemx cl_F ftruncate (const @var{type}& x, const @var{type}& y)
@itemx cl_F fround (const @var{type}& x, const @var{type}& y)
@end table

These functions also exist in versions which return both the quotient
and the remainder. The suffix @samp{2} indicates this.

Each of the classes
@code{cl_F}, @code{cl_SF}, @code{cl_FF}, @code{cl_DF}, @code{cl_LF}
defines the following operations:
@cindex @code{cl_F_fdiv_t}
@cindex @code{cl_SF_fdiv_t}
@cindex @code{cl_FF_fdiv_t}
@cindex @code{cl_DF_fdiv_t}
@cindex @code{cl_LF_fdiv_t}

@table @code
@item struct @var{type}_fdiv_t @{ @var{type} quotient; @var{type} remainder; @};
@itemx @var{type}_fdiv_t ffloor2 (const @var{type}& x)
@cindex @code{ffloor2 ()}
@itemx @var{type}_fdiv_t fceiling2 (const @var{type}& x)
@cindex @code{fceiling2 ()}
@itemx @var{type}_fdiv_t ftruncate2 (const @var{type}& x)
@cindex @code{ftruncate2 ()}
@itemx @var{type}_fdiv_t fround2 (const @var{type}& x)
@cindex @code{fround2 ()}
@end table
and similarly for class @code{cl_R}, but with quotient type @code{cl_F}.
@cindex @code{cl_R_fdiv_t}

The class @code{cl_R} defines the following operations:

@table @code
@item struct @var{type}_fdiv_t @{ cl_F quotient; cl_R remainder; @};
@itemx @var{type}_fdiv_t ffloor2 (const @var{type}& x, const @var{type}& y)
@itemx @var{type}_fdiv_t fceiling2 (const @var{type}& x, const @var{type}& y)
@itemx @var{type}_fdiv_t ftruncate2 (const @var{type}& x, const @var{type}& y)
@itemx @var{type}_fdiv_t fround2 (const @var{type}& x, const @var{type}& y)
@end table

Other applications need only the remainder of a division.
The remainder of @samp{floor} and @samp{ffloor} is called @samp{mod}
(abbreviation of ``modulo''). The remainder @samp{truncate} and
@samp{ftruncate} is called @samp{rem} (abbreviation of ``remainder'').

@itemize @bullet
@item
@code{mod(x,y) = floor2(x,y).remainder = x - floor(x/y)*y}
@item
@code{rem(x,y) = truncate2(x,y).remainder = x - truncate(x/y)*y}
@end itemize

If @code{x} and @code{y} are both >= 0, @code{mod(x,y) = rem(x,y) >= 0}.
In general, @code{mod(x,y)} has the sign of @code{y} or is zero,
and @code{rem(x,y)} has the sign of @code{x} or is zero.

The classes @code{cl_R}, @code{cl_I} define the following operations:

@table @code
@item @var{type} mod (const @var{type}& x, const @var{type}& y)
@cindex @code{mod ()}
@itemx @var{type} rem (const @var{type}& x, const @var{type}& y)
@cindex @code{rem ()}
@end table


@section Roots

Each of the classes @code{cl_R},
@code{cl_F}, @code{cl_SF}, @code{cl_FF}, @code{cl_DF}, @code{cl_LF}
defines the following operation:

@table @code
@item @var{type} sqrt (const @var{type}& x)
@cindex @code{sqrt ()}
@code{x} must be >= 0. This function returns the square root of @code{x},
normalized to be >= 0. If @code{x} is the square of a rational number,
@code{sqrt(x)} will be a rational number, else it will return a
floating-point approximation.
@end table

The classes @code{cl_RA}, @code{cl_I} define the following operation:

@table @code
@item cl_boolean sqrtp (const @var{type}& x, @var{type}* root)
@cindex @code{sqrtp ()}
This tests whether @code{x} is a perfect square. If so, it returns true
and the exact square root in @code{*root}, else it returns false.
@end table

Furthermore, for integers, similarly:

@table @code
@item cl_boolean isqrt (const @var{type}& x, @var{type}* root)
@cindex @code{isqrt ()}
@code{x} should be >= 0. This function sets @code{*root} to
@code{floor(sqrt(x))} and returns the same value as @code{sqrtp}:
the boolean value @code{(expt(*root,2) == x)}.
@end table

For @code{n}th roots, the classes @code{cl_RA}, @code{cl_I}
define the following operation:

@table @code
@item cl_boolean rootp (const @var{type}& x, const cl_I& n, @var{type}* root)
@cindex @code{rootp ()}
@code{x} must be >= 0. @code{n} must be > 0.
This tests whether @code{x} is an @code{n}th power of a rational number.
If so, it returns true and the exact root in @code{*root}, else it returns
false.
@end table

The only square root function which accepts negative numbers is the one
for class @code{cl_N}:

@table @code
@item cl_N sqrt (const cl_N& z)
@cindex @code{sqrt ()}
Returns the square root of @code{z}, as defined by the formula
@code{sqrt(z) = exp(log(z)/2)}. Conversion to a floating-point type
or to a complex number are done if necessary. The range of the result is the
right half plane @code{realpart(sqrt(z)) >= 0}
including the positive imaginary axis and 0, but excluding
the negative imaginary axis.
The result is an exact number only if @code{z} is an exact number.
@end table


@section Transcendental functions
@cindex transcendental functions

The transcendental functions return an exact result if the argument
is exact and the result is exact as well. Otherwise they must return
inexact numbers even if the argument is exact.
For example, @code{cos(0) = 1} returns the rational number @code{1}.


@subsection Exponential and logarithmic functions

@table @code
@item cl_R exp (const cl_R& x)
@cindex @code{exp ()}
@itemx cl_N exp (const cl_N& x)
Returns the exponential function of @code{x}. This is @code{e^x} where
@code{e} is the base of the natural logarithms. The range of the result
is the entire complex plane excluding 0.

@item cl_R ln (const cl_R& x)
@cindex @code{ln ()}
@code{x} must be > 0. Returns the (natural) logarithm of x.

@item cl_N log (const cl_N& x)
@cindex @code{log ()}
Returns the (natural) logarithm of x. If @code{x} is real and positive,
this is @code{ln(x)}. In general, @code{log(x) = log(abs(x)) + i*phase(x)}.
The range of the result is the strip in the complex plane
@code{-pi < imagpart(log(x)) <= pi}.

@item cl_R phase (const cl_N& x)
@cindex @code{phase ()}
Returns the angle part of @code{x} in its polar representation as a
complex number. That is, @code{phase(x) = atan(realpart(x),imagpart(x))}.
This is also the imaginary part of @code{log(x)}.
The range of the result is the interval @code{-pi < phase(x) <= pi}.
The result will be an exact number only if @code{zerop(x)} or
if @code{x} is real and positive.

@item cl_R log (const cl_R& a, const cl_R& b)
@code{a} and @code{b} must be > 0. Returns the logarithm of @code{a} with
respect to base @code{b}. @code{log(a,b) = ln(a)/ln(b)}.
The result can be exact only if @code{a = 1} or if @code{a} and @code{b}
are both rational.

@item cl_N log (const cl_N& a, const cl_N& b)
Returns the logarithm of @code{a} with respect to base @code{b}.
@code{log(a,b) = log(a)/log(b)}.

@item cl_N expt (const cl_N& x, const cl_N& y)
@cindex @code{expt ()}
Exponentiation: Returns @code{x^y = exp(y*log(x))}.
@end table

The constant e = exp(1) = 2.71828@dots{} is returned by the following functions:

@table @code
@item cl_F exp1 (float_format_t f)
@cindex @code{exp1 ()}
Returns e as a float of format @code{f}.

@item cl_F exp1 (const cl_F& y)
Returns e in the float format of @code{y}.

@item cl_F exp1 (void)
Returns e as a float of format @code{default_float_format}.
@end table


@subsection Trigonometric functions

@table @code
@item cl_R sin (const cl_R& x)
@cindex @code{sin ()}
Returns @code{sin(x)}. The range of the result is the interval
@code{-1 <= sin(x) <= 1}.

@item cl_N sin (const cl_N& z)
Returns @code{sin(z)}. The range of the result is the entire complex plane.

@item cl_R cos (const cl_R& x)
@cindex @code{cos ()}
Returns @code{cos(x)}. The range of the result is the interval
@code{-1 <= cos(x) <= 1}.

@item cl_N cos (const cl_N& x)
Returns @code{cos(z)}. The range of the result is the entire complex plane.

@item struct cos_sin_t @{ cl_R cos; cl_R sin; @};
@cindex @code{cos_sin_t}
@itemx cos_sin_t cos_sin (const cl_R& x)
Returns both @code{sin(x)} and @code{cos(x)}. This is more efficient than
@cindex @code{cos_sin ()}
computing them separately. The relation @code{cos^2 + sin^2 = 1} will
hold only approximately.

@item cl_R tan (const cl_R& x)
@cindex @code{tan ()}
@itemx cl_N tan (const cl_N& x)
Returns @code{tan(x) = sin(x)/cos(x)}.

@item cl_N cis (const cl_R& x)
@cindex @code{cis ()}
@itemx cl_N cis (const cl_N& x)
Returns @code{exp(i*x)}. The name @samp{cis} means ``cos + i sin'', because
@code{e^(i*x) = cos(x) + i*sin(x)}.

@cindex @code{asin}
@cindex @code{asin ()}
@item cl_N asin (const cl_N& z)
Returns @code{arcsin(z)}. This is defined as
@code{arcsin(z) = log(iz+sqrt(1-z^2))/i} and satisfies
@code{arcsin(-z) = -arcsin(z)}.
The range of the result is the strip in the complex domain
@code{-pi/2 <= realpart(arcsin(z)) <= pi/2}, excluding the numbers
with @code{realpart = -pi/2} and @code{imagpart < 0} and the numbers
with @code{realpart = pi/2} and @code{imagpart > 0}.
@ignore
Proof: This follows from arcsin(z) = arsinh(iz)/i and the corresponding
results for arsinh.
@end ignore

@item cl_N acos (const cl_N& z)
@cindex @code{acos ()}
Returns @code{arccos(z)}. This is defined as
@code{arccos(z) = pi/2 - arcsin(z) = log(z+i*sqrt(1-z^2))/i}
@ignore
 Kahan's formula:
 @code{arccos(z) = 2*log(sqrt((1+z)/2)+i*sqrt((1-z)/2))/i}
@end ignore
and satisfies @code{arccos(-z) = pi - arccos(z)}.
The range of the result is the strip in the complex domain
@code{0 <= realpart(arcsin(z)) <= pi}, excluding the numbers
with @code{realpart = 0} and @code{imagpart < 0} and the numbers
with @code{realpart = pi} and @code{imagpart > 0}.
@ignore
Proof: This follows from the results about arcsin.
@end ignore

@cindex @code{atan}
@cindex @code{atan ()}
@item cl_R atan (const cl_R& x, const cl_R& y)
Returns the angle of the polar representation of the complex number
@code{x+iy}. This is @code{atan(y/x)} if @code{x>0}. The range of
the result is the interval @code{-pi < atan(x,y) <= pi}. The result will
be an exact number only if @code{x > 0} and @code{y} is the exact @code{0}.
WARNING: In Common Lisp, this function is called as @code{(atan y x)},
with reversed order of arguments.

@item cl_R atan (const cl_R& x)
Returns @code{arctan(x)}. This is the same as @code{atan(1,x)}. The range
of the result is the interval @code{-pi/2 < atan(x) < pi/2}. The result
will be an exact number only if @code{x} is the exact @code{0}.

@item cl_N atan (const cl_N& z)
Returns @code{arctan(z)}. This is defined as
@code{arctan(z) = (log(1+iz)-log(1-iz)) / 2i} and satisfies
@code{arctan(-z) = -arctan(z)}. The range of the result is
the strip in the complex domain
@code{-pi/2 <= realpart(arctan(z)) <= pi/2}, excluding the numbers
with @code{realpart = -pi/2} and @code{imagpart >= 0} and the numbers
with @code{realpart = pi/2} and @code{imagpart <= 0}.
@ignore
Proof: arctan(z) = artanh(iz)/i, we know the range of the artanh function.
@end ignore

@end table

@cindex pi
@cindex Archimedes' constant
Archimedes' constant pi = 3.14@dots{} is returned by the following functions:

@table @code
@item cl_F pi (float_format_t f)
@cindex @code{pi ()}
Returns pi as a float of format @code{f}.

@item cl_F pi (const cl_F& y)
Returns pi in the float format of @code{y}.

@item cl_F pi (void)
Returns pi as a float of format @code{default_float_format}.
@end table


@subsection Hyperbolic functions

@table @code
@item cl_R sinh (const cl_R& x)
@cindex @code{sinh ()}
Returns @code{sinh(x)}.

@item cl_N sinh (const cl_N& z)
Returns @code{sinh(z)}. The range of the result is the entire complex plane.

@item cl_R cosh (const cl_R& x)
@cindex @code{cosh ()}
Returns @code{cosh(x)}. The range of the result is the interval
@code{cosh(x) >= 1}.

@item cl_N cosh (const cl_N& z)
Returns @code{cosh(z)}. The range of the result is the entire complex plane.

@item struct cosh_sinh_t @{ cl_R cosh; cl_R sinh; @};
@cindex @code{cosh_sinh_t}
@itemx cosh_sinh_t cosh_sinh (const cl_R& x)
@cindex @code{cosh_sinh ()}
Returns both @code{sinh(x)} and @code{cosh(x)}. This is more efficient than
computing them separately. The relation @code{cosh^2 - sinh^2 = 1} will
hold only approximately.

@item cl_R tanh (const cl_R& x)
@cindex @code{tanh ()}
@itemx cl_N tanh (const cl_N& x)
Returns @code{tanh(x) = sinh(x)/cosh(x)}.

@item cl_N asinh (const cl_N& z)
@cindex @code{asinh ()}
Returns @code{arsinh(z)}. This is defined as
@code{arsinh(z) = log(z+sqrt(1+z^2))} and satisfies
@code{arsinh(-z) = -arsinh(z)}.
@ignore
Proof: Knowing the range of log, we know -pi < imagpart(arsinh(z)) <= pi.
Actually, z+sqrt(1+z^2) can never be real and <0, so
-pi < imagpart(arsinh(z)) < pi.
We have (z+sqrt(1+z^2))*(-z+sqrt(1+(-z)^2)) = (1+z^2)-z^2 = 1, hence the
logs of both factors sum up to 0 mod 2*pi*i, hence to 0.
@end ignore
The range of the result is the strip in the complex domain
@code{-pi/2 <= imagpart(arsinh(z)) <= pi/2}, excluding the numbers
with @code{imagpart = -pi/2} and @code{realpart > 0} and the numbers
with @code{imagpart = pi/2} and @code{realpart < 0}.
@ignore
Proof: Write z = x+iy. Because of arsinh(-z) = -arsinh(z), we may assume
that z is in Range(sqrt), that is, x>=0 and, if x=0, then y>=0.
If x > 0, then Re(z+sqrt(1+z^2)) = x + Re(sqrt(1+z^2)) >= x > 0,
so -pi/2 < imagpart(log(z+sqrt(1+z^2))) < pi/2.
If x = 0 and y >= 0, arsinh(z) = log(i*y+sqrt(1-y^2)).
  If y <= 1, the realpart is 0 and the imagpart is >= 0 and <= pi/2.
  If y >= 1, the imagpart is pi/2 and the realpart is
             log(y+sqrt(y^2-1)) >= log(y) >= 0.
@end ignore
@ignore
Moreover, if z is in Range(sqrt),
log(sqrt(1+z^2)+z) = 2 artanh(z/(1+sqrt(1+z^2)))
(for a proof, see file src/cl_C_asinh.cc).
@end ignore

@item cl_N acosh (const cl_N& z)
@cindex @code{acosh ()}
Returns @code{arcosh(z)}. This is defined as
@code{arcosh(z) = 2*log(sqrt((z+1)/2)+sqrt((z-1)/2))}.
The range of the result is the half-strip in the complex domain
@code{-pi < imagpart(arcosh(z)) <= pi, realpart(arcosh(z)) >= 0},
excluding the numbers with @code{realpart = 0} and @code{-pi < imagpart < 0}.
@ignore
Proof: sqrt((z+1)/2) and sqrt((z-1)/2)) lie in Range(sqrt), hence does
their sum, hence its log has an imagpart <= pi/2 and > -pi/2.
If z is in Range(sqrt), we have
  sqrt(z+1)*sqrt(z-1) = sqrt(z^2-1)
  ==> (sqrt((z+1)/2)+sqrt((z-1)/2))^2 = (z+1)/2 + sqrt(z^2-1) + (z-1)/2
                                      = z + sqrt(z^2-1)
  ==> arcosh(z) = log(z+sqrt(z^2-1)) mod 2*pi*i
  and since the imagpart of both expressions is > -pi, <= pi
  ==> arcosh(z) = log(z+sqrt(z^2-1))
  To prove that the realpart of this is >= 0, write z = x+iy with x>=0,
  z^2-1 = u+iv with u = x^2-y^2-1, v = 2xy,
  sqrt(z^2-1) = p+iq with p = sqrt((sqrt(u^2+v^2)+u)/2) >= 0,
                          q = sqrt((sqrt(u^2+v^2)-u)/2) * sign(v),
  then |z+sqrt(z^2-1)|^2 = |x+iy + p+iq|^2
          = (x+p)^2 + (y+q)^2
          = x^2 + 2xp + p^2 + y^2 + 2yq + q^2
          >= x^2 + p^2 + y^2 + q^2                 (since x>=0, p>=0, yq>=0)
          = x^2 + y^2 + sqrt(u^2+v^2)
          >= x^2 + y^2 + |u|
          >= x^2 + y^2 - u
          = 1 + 2*y^2
          >= 1
  hence realpart(log(z+sqrt(z^2-1))) = log(|z+sqrt(z^2-1)|) >= 0.
  Equality holds only if y = 0 and u <= 0, i.e. 0 <= x < 1.
  In this case arcosh(z) = log(x+i*sqrt(1-x^2)) has imagpart >=0.
Otherwise, -z is in Range(sqrt).
  If y != 0, sqrt((z+1)/2) = i^sign(y) * sqrt((-z-1)/2),
             sqrt((z-1)/2) = i^sign(y) * sqrt((-z+1)/2),
             hence arcosh(z) = sign(y)*pi/2*i + arcosh(-z),
             and this has realpart > 0.
  If y = 0 and -1<=x<=0, we still have sqrt(z+1)*sqrt(z-1) = sqrt(z^2-1),
             ==> arcosh(z) = log(z+sqrt(z^2-1)) = log(x+i*sqrt(1-x^2))
             has realpart = 0 and imagpart > 0.
  If y = 0 and x<=-1, however, sqrt(z+1)*sqrt(z-1) = - sqrt(z^2-1),
             ==> arcosh(z) = log(z-sqrt(z^2-1)) = pi*i + arcosh(-z).
             This has realpart >= 0 and imagpart = pi.
@end ignore

@item cl_N atanh (const cl_N& z)
@cindex @code{atanh ()}
Returns @code{artanh(z)}. This is defined as
@code{artanh(z) = (log(1+z)-log(1-z)) / 2} and satisfies
@code{artanh(-z) = -artanh(z)}. The range of the result is
the strip in the complex domain
@code{-pi/2 <= imagpart(artanh(z)) <= pi/2}, excluding the numbers
with @code{imagpart = -pi/2} and @code{realpart <= 0} and the numbers
with @code{imagpart = pi/2} and @code{realpart >= 0}.
@ignore
Proof: Write z = x+iy. Examine
  imagpart(artanh(z)) = (atan(1+x,y) - atan(1-x,-y))/2.
  Case 1: y = 0.
          x > 1 ==> imagpart = -pi/2, realpart = 1/2 log((x+1)/(x-1)) > 0,
          x < -1 ==> imagpart = pi/2, realpart = 1/2 log((-x-1)/(-x+1)) < 0,
          |x| < 1 ==> imagpart = 0
  Case 2: y > 0.
          imagpart(artanh(z))
              = (atan(1+x,y) - atan(1-x,-y))/2
              = ((pi/2 - atan((1+x)/y)) - (-pi/2 - atan((1-x)/-y)))/2
              = (pi - atan((1+x)/y) - atan((1-x)/y))/2
              > (pi -     pi/2      -     pi/2     )/2 = 0
          and (1+x)/y > (1-x)/y
              ==> atan((1+x)/y) > atan((-1+x)/y) = - atan((1-x)/y)
              ==> imagpart < pi/2.
          Hence 0 < imagpart < pi/2.
  Case 3: y < 0.
          By artanh(z) = -artanh(-z) and case 2, -pi/2 < imagpart < 0.
@end ignore
@end table


@subsection Euler gamma
@cindex Euler's constant

Euler's constant C = 0.577@dots{} is returned by the following functions:

@table @code
@item cl_F eulerconst (float_format_t f)
@cindex @code{eulerconst ()}
Returns Euler's constant as a float of format @code{f}.

@item cl_F eulerconst (const cl_F& y)
Returns Euler's constant in the float format of @code{y}.

@item cl_F eulerconst (void)
Returns Euler's constant as a float of format @code{default_float_format}.
@end table

Catalan's constant G = 0.915@dots{} is returned by the following functions:
@cindex Catalan's constant

@table @code
@item cl_F catalanconst (float_format_t f)
@cindex @code{catalanconst ()}
Returns Catalan's constant as a float of format @code{f}.

@item cl_F catalanconst (const cl_F& y)
Returns Catalan's constant in the float format of @code{y}.

@item cl_F catalanconst (void)
Returns Catalan's constant as a float of format @code{default_float_format}.
@end table


@subsection Riemann zeta
@cindex Riemann's zeta

Riemann's zeta function at an integral point @code{s>1} is returned by the
following functions:

@table @code
@item cl_F zeta (int s, float_format_t f)
@cindex @code{zeta ()}
Returns Riemann's zeta function at @code{s} as a float of format @code{f}.

@item cl_F zeta (int s, const cl_F& y)
Returns Riemann's zeta function at @code{s} in the float format of @code{y}.

@item cl_F zeta (int s)
Returns Riemann's zeta function at @code{s} as a float of format
@code{default_float_format}.
@end table


@section Functions on integers

@subsection Logical functions

Integers, when viewed as in two's complement notation, can be thought as
infinite bit strings where the bits' values eventually are constant.
For example,
@example
    17 = ......00010001
    -6 = ......11111010
@end example

The logical operations view integers as such bit strings and operate
on each of the bit positions in parallel.

@table @code
@item cl_I lognot (const cl_I& x)
@cindex @code{lognot ()}
@itemx cl_I operator ~ (const cl_I& x)
@cindex @code{operator ~ ()}
Logical not, like @code{~x} in C. This is the same as @code{-1-x}.

@item cl_I logand (const cl_I& x, const cl_I& y)
@cindex @code{logand ()}
@itemx cl_I operator & (const cl_I& x, const cl_I& y)
@cindex @code{operator & ()}
Logical and, like @code{x & y} in C.

@item cl_I logior (const cl_I& x, const cl_I& y)
@cindex @code{logior ()}
@itemx cl_I operator | (const cl_I& x, const cl_I& y)
@cindex @code{operator | ()}
Logical (inclusive) or, like @code{x | y} in C.

@item cl_I logxor (const cl_I& x, const cl_I& y)
@cindex @code{logxor ()}
@itemx cl_I operator ^ (const cl_I& x, const cl_I& y)
@cindex @code{operator ^ ()}
Exclusive or, like @code{x ^ y} in C.

@item cl_I logeqv (const cl_I& x, const cl_I& y)
@cindex @code{logeqv ()}
Bitwise equivalence, like @code{~(x ^ y)} in C.

@item cl_I lognand (const cl_I& x, const cl_I& y)
@cindex @code{lognand ()}
Bitwise not and, like @code{~(x & y)} in C.

@item cl_I lognor (const cl_I& x, const cl_I& y)
@cindex @code{lognor ()}
Bitwise not or, like @code{~(x | y)} in C.

@item cl_I logandc1 (const cl_I& x, const cl_I& y)
@cindex @code{logandc1 ()}
Logical and, complementing the first argument, like @code{~x & y} in C.

@item cl_I logandc2 (const cl_I& x, const cl_I& y)
@cindex @code{logandc2 ()}
Logical and, complementing the second argument, like @code{x & ~y} in C.

@item cl_I logorc1 (const cl_I& x, const cl_I& y)
@cindex @code{logorc1 ()}
Logical or, complementing the first argument, like @code{~x | y} in C.

@item cl_I logorc2 (const cl_I& x, const cl_I& y)
@cindex @code{logorc2 ()}
Logical or, complementing the second argument, like @code{x | ~y} in C.
@end table

These operations are all available though the function
@table @code
@item cl_I boole (cl_boole op, const cl_I& x, const cl_I& y)
@cindex @code{boole ()}
@end table
where @code{op} must have one of the 16 values (each one stands for a function
which combines two bits into one bit): @code{boole_clr}, @code{boole_set},
@code{boole_1}, @code{boole_2}, @code{boole_c1}, @code{boole_c2},
@code{boole_and}, @code{boole_ior}, @code{boole_xor}, @code{boole_eqv},
@code{boole_nand}, @code{boole_nor}, @code{boole_andc1}, @code{boole_andc2},
@code{boole_orc1}, @code{boole_orc2}.
@cindex @code{boole_clr}
@cindex @code{boole_set}
@cindex @code{boole_1}
@cindex @code{boole_2}
@cindex @code{boole_c1}
@cindex @code{boole_c2}
@cindex @code{boole_and}
@cindex @code{boole_xor}
@cindex @code{boole_eqv}
@cindex @code{boole_nand}
@cindex @code{boole_nor}
@cindex @code{boole_andc1}
@cindex @code{boole_andc2}
@cindex @code{boole_orc1}
@cindex @code{boole_orc2}


Other functions that view integers as bit strings:

@table @code
@item cl_boolean logtest (const cl_I& x, const cl_I& y)
@cindex @code{logtest ()}
Returns true if some bit is set in both @code{x} and @code{y}, i.e. if
@code{logand(x,y) != 0}.

@item cl_boolean logbitp (const cl_I& n, const cl_I& x)
@cindex @code{logbitp ()}
Returns true if the @code{n}th bit (from the right) of @code{x} is set.
Bit 0 is the least significant bit.

@item uintL logcount (const cl_I& x)
@cindex @code{logcount ()}
Returns the number of one bits in @code{x}, if @code{x} >= 0, or
the number of zero bits in @code{x}, if @code{x} < 0.
@end table

The following functions operate on intervals of bits in integers. 
The type
@example
struct cl_byte @{ uintL size; uintL position; @};
@end example
@cindex @code{cl_byte}
represents the bit interval containing the bits
@code{position}@dots{}@code{position+size-1} of an integer.
The constructor @code{cl_byte(size,position)} constructs a @code{cl_byte}.

@table @code
@item cl_I ldb (const cl_I& n, const cl_byte& b)
@cindex @code{ldb ()}
extracts the bits of @code{n} described by the bit interval @code{b}
and returns them as a nonnegative integer with @code{b.size} bits.

@item cl_boolean ldb_test (const cl_I& n, const cl_byte& b)
@cindex @code{ldb_test ()}
Returns true if some bit described by the bit interval @code{b} is set in
@code{n}.

@item cl_I dpb (const cl_I& newbyte, const cl_I& n, const cl_byte& b)
@cindex @code{dpb ()}
Returns @code{n}, with the bits described by the bit interval @code{b}
replaced by @code{newbyte}. Only the lowest @code{b.size} bits of
@code{newbyte} are relevant.
@end table

The functions @code{ldb} and @code{dpb} implicitly shift. The following
functions are their counterparts without shifting:

@table @code
@item cl_I mask_field (const cl_I& n, const cl_byte& b)
@cindex @code{mask_field ()}
returns an integer with the bits described by the bit interval @code{b}
copied from the corresponding bits in @code{n}, the other bits zero.

@item cl_I deposit_field (const cl_I& newbyte, const cl_I& n, const cl_byte& b)
@cindex @code{deposit_field ()}
returns an integer where the bits described by the bit interval @code{b}
come from @code{newbyte} and the other bits come from @code{n}.
@end table

The following relations hold:

@itemize @asis
@item
@code{ldb (n, b) = mask_field(n, b) >> b.position},
@item
@code{dpb (newbyte, n, b) = deposit_field (newbyte << b.position, n, b)},
@item
@code{deposit_field(newbyte,n,b) = n ^ mask_field(n,b) ^ mask_field(new_byte,b)}.
@end itemize

The following operations on integers as bit strings are efficient shortcuts
for common arithmetic operations:

@table @code
@item cl_boolean oddp (const cl_I& x)
@cindex @code{oddp ()}
Returns true if the least significant bit of @code{x} is 1. Equivalent to
@code{mod(x,2) != 0}.

@item cl_boolean evenp (const cl_I& x)
@cindex @code{evenp ()}
Returns true if the least significant bit of @code{x} is 0. Equivalent to
@code{mod(x,2) == 0}.

@item cl_I operator << (const cl_I& x, const cl_I& n)
@cindex @code{operator << ()}
Shifts @code{x} by @code{n} bits to the left. @code{n} should be >=0.
Equivalent to @code{x * expt(2,n)}.

@item cl_I operator >> (const cl_I& x, const cl_I& n)
@cindex @code{operator >> ()}
Shifts @code{x} by @code{n} bits to the right. @code{n} should be >=0.
Bits shifted out to the right are thrown away.
Equivalent to @code{floor(x / expt(2,n))}.

@item cl_I ash (const cl_I& x, const cl_I& y)
@cindex @code{ash ()}
Shifts @code{x} by @code{y} bits to the left (if @code{y}>=0) or
by @code{-y} bits to the right (if @code{y}<=0). In other words, this
returns @code{floor(x * expt(2,y))}.

@item uintL integer_length (const cl_I& x)
@cindex @code{integer_length ()}
Returns the number of bits (excluding the sign bit) needed to represent @code{x}
in two's complement notation. This is the smallest n >= 0 such that
-2^n <= x < 2^n. If x > 0, this is the unique n > 0 such that
2^(n-1) <= x < 2^n.

@item uintL ord2 (const cl_I& x)
@cindex @code{ord2 ()}
@code{x} must be non-zero. This function returns the number of 0 bits at the
right of @code{x} in two's complement notation. This is the largest n >= 0
such that 2^n divides @code{x}.

@item uintL power2p (const cl_I& x)
@cindex @code{power2p ()}
@code{x} must be > 0. This function checks whether @code{x} is a power of 2.
If @code{x} = 2^(n-1), it returns n. Else it returns 0.
(See also the function @code{logp}.)
@end table


@subsection Number theoretic functions

@table @code
@item uint32 gcd (uint32 a, uint32 b)
@cindex @code{gcd ()}
@itemx cl_I gcd (const cl_I& a, const cl_I& b)
This function returns the greatest common divisor of @code{a} and @code{b},
normalized to be >= 0.

@item cl_I xgcd (const cl_I& a, const cl_I& b, cl_I* u, cl_I* v)
@cindex @code{xgcd ()}
This function (``extended gcd'') returns the greatest common divisor @code{g} of
@code{a} and @code{b} and at the same time the representation of @code{g}
as an integral linear combination of @code{a} and @code{b}:
@code{u} and @code{v} with @code{u*a+v*b = g}, @code{g} >= 0.
@code{u} and @code{v} will be normalized to be of smallest possible absolute
value, in the following sense: If @code{a} and @code{b} are non-zero, and
@code{abs(a) != abs(b)}, @code{u} and @code{v} will satisfy the inequalities
@code{abs(u) <= abs(b)/(2*g)}, @code{abs(v) <= abs(a)/(2*g)}.

@item cl_I lcm (const cl_I& a, const cl_I& b)
@cindex @code{lcm ()}
This function returns the least common multiple of @code{a} and @code{b},
normalized to be >= 0.

@item cl_boolean logp (const cl_I& a, const cl_I& b, cl_RA* l)
@cindex @code{logp ()}
@itemx cl_boolean logp (const cl_RA& a, const cl_RA& b, cl_RA* l)
@code{a} must be > 0. @code{b} must be >0 and != 1. If log(a,b) is
rational number, this function returns true and sets *l = log(a,b), else
it returns false.

@item int jacobi (sint32 a, sint32 b)
@cindex @code{jacobi()}
@itemx int jacobi (const cl_I& a, const cl_I& b)
Returns the Jacobi symbol 
@tex 
$\left({a\over b}\right)$,
@end tex
@ifnottex 
(a/b),
@end ifnottex
@code{a,b} must be integers, @code{b>0} and odd. The result is 0
iff gcd(a,b)>1.

@item cl_boolean isprobprime (const cl_I& n)
@cindex prime
@cindex @code{isprobprime()}
Returns true if @code{n} is a small prime or passes the Miller-Rabin 
primality test. The probability of a false positive is 1:10^30.

@item cl_I nextprobprime (const cl_R& x)
@cindex @code{nextprobprime()}
Returns the smallest probable prime >=@code{x}.
@end table


@subsection Combinatorial functions

@table @code
@item cl_I factorial (uintL n)
@cindex @code{factorial ()}
@code{n} must be a small integer >= 0. This function returns the factorial
@code{n}! = @code{1*2*@dots{}*n}.

@item cl_I doublefactorial (uintL n)
@cindex @code{doublefactorial ()}
@code{n} must be a small integer >= 0. This function returns the 
doublefactorial @code{n}!! = @code{1*3*@dots{}*n} or 
@code{n}!! = @code{2*4*@dots{}*n}, respectively.

@item cl_I binomial (uintL n, uintL k)
@cindex @code{binomial ()}
@code{n} and @code{k} must be small integers >= 0. This function returns the
binomial coefficient
@tex
${n \choose k} = {n! \over n! (n-k)!}$
@end tex
@ifinfo
(@code{n} choose @code{k}) = @code{n}! / @code{k}! @code{(n-k)}!
@end ifinfo
for 0 <= k <= n, 0 else.
@end table


@section Functions on floating-point numbers

Recall that a floating-point number consists of a sign @code{s}, an
exponent @code{e} and a mantissa @code{m}. The value of the number is
@code{(-1)^s * 2^e * m}.

Each of the classes
@code{cl_F}, @code{cl_SF}, @code{cl_FF}, @code{cl_DF}, @code{cl_LF}
defines the following operations.

@table @code
@item @var{type} scale_float (const @var{type}& x, sintL delta)
@cindex @code{scale_float ()}
@itemx @var{type} scale_float (const @var{type}& x, const cl_I& delta)
Returns @code{x*2^delta}. This is more efficient than an explicit multiplication
because it copies @code{x} and modifies the exponent.
@end table

The following functions provide an abstract interface to the underlying
representation of floating-point numbers.

@table @code
@item sintL float_exponent (const @var{type}& x)
@cindex @code{float_exponent ()}
Returns the exponent @code{e} of @code{x}.
For @code{x = 0.0}, this is 0. For @code{x} non-zero, this is the unique
integer with @code{2^(e-1) <= abs(x) < 2^e}.

@item sintL float_radix (const @var{type}& x)
@cindex @code{float_radix ()}
Returns the base of the floating-point representation. This is always @code{2}.

@item @var{type} float_sign (const @var{type}& x)
@cindex @code{float_sign ()}
Returns the sign @code{s} of @code{x} as a float. The value is 1 for
@code{x} >= 0, -1 for @code{x} < 0.

@item uintL float_digits (const @var{type}& x)
@cindex @code{float_digits ()}
Returns the number of mantissa bits in the floating-point representation
of @code{x}, including the hidden bit. The value only depends on the type
of @code{x}, not on its value.

@item uintL float_precision (const @var{type}& x)
@cindex @code{float_precision ()}
Returns the number of significant mantissa bits in the floating-point
representation of @code{x}. Since denormalized numbers are not supported,
this is the same as @code{float_digits(x)} if @code{x} is non-zero, and
0 if @code{x} = 0.
@end table

The complete internal representation of a float is encoded in the type
@cindex @code{decoded_float}
@cindex @code{decoded_sfloat}
@cindex @code{decoded_ffloat}
@cindex @code{decoded_dfloat}
@cindex @code{decoded_lfloat}
@code{decoded_float} (or @code{decoded_sfloat}, @code{decoded_ffloat},
@code{decoded_dfloat}, @code{decoded_lfloat}, respectively), defined by
@example
struct decoded_@var{type}float @{
        @var{type} mantissa; cl_I exponent; @var{type} sign;
@};
@end example

and returned by the function

@table @code
@item decoded_@var{type}float decode_float (const @var{type}& x)
@cindex @code{decode_float ()}
For @code{x} non-zero, this returns @code{(-1)^s}, @code{e}, @code{m} with
@code{x = (-1)^s * 2^e * m} and @code{0.5 <= m < 1.0}. For @code{x} = 0,
it returns @code{(-1)^s}=1, @code{e}=0, @code{m}=0.
@code{e} is the same as returned by the function @code{float_exponent}.
@end table

A complete decoding in terms of integers is provided as type
@cindex @code{cl_idecoded_float}
@example
struct cl_idecoded_float @{
        cl_I mantissa; cl_I exponent; cl_I sign;
@};
@end example
by the following function:

@table @code
@item cl_idecoded_float integer_decode_float (const @var{type}& x)
@cindex @code{integer_decode_float ()}
For @code{x} non-zero, this returns @code{(-1)^s}, @code{e}, @code{m} with
@code{x = (-1)^s * 2^e * m} and @code{m} an integer with @code{float_digits(x)}
bits. For @code{x} = 0, it returns @code{(-1)^s}=1, @code{e}=0, @code{m}=0.
WARNING: The exponent @code{e} is not the same as the one returned by
the functions @code{decode_float} and @code{float_exponent}.
@end table

Some other function, implemented only for class @code{cl_F}:

@table @code
@item cl_F float_sign (const cl_F& x, const cl_F& y)
@cindex @code{float_sign ()}
This returns a floating point number whose precision and absolute value
is that of @code{y} and whose sign is that of @code{x}. If @code{x} is
zero, it is treated as positive. Same for @code{y}.
@end table


@section Conversion functions
@cindex conversion

@subsection Conversion to floating-point numbers

The type @code{float_format_t} describes a floating-point format.
@cindex @code{float_format_t}

@table @code
@item float_format_t float_format (uintL n)
@cindex @code{float_format ()}
Returns the smallest float format which guarantees at least @code{n}
decimal digits in the mantissa (after the decimal point).

@item float_format_t float_format (const cl_F& x)
Returns the floating point format of @code{x}.

@item float_format_t default_float_format
@cindex @code{default_float_format}
Global variable: the default float format used when converting rational numbers
to floats.
@end table

To convert a real number to a float, each of the types
@code{cl_R}, @code{cl_F}, @code{cl_I}, @code{cl_RA},
@code{int}, @code{unsigned int}, @code{float}, @code{double}
defines the following operations:

@table @code
@item cl_F cl_float (const @var{type}&x, float_format_t f)
@cindex @code{cl_float ()}
Returns @code{x} as a float of format @code{f}.
@item cl_F cl_float (const @var{type}&x, const cl_F& y)
Returns @code{x} in the float format of @code{y}.
@item cl_F cl_float (const @var{type}&x)
Returns @code{x} as a float of format @code{default_float_format} if
it is an exact number, or @code{x} itself if it is already a float.
@end table

Of course, converting a number to a float can lose precision.

Every floating-point format has some characteristic numbers:

@table @code
@item cl_F most_positive_float (float_format_t f)
@cindex @code{most_positive_float ()}
Returns the largest (most positive) floating point number in float format @code{f}.

@item cl_F most_negative_float (float_format_t f)
@cindex @code{most_negative_float ()}
Returns the smallest (most negative) floating point number in float format @code{f}.

@item cl_F least_positive_float (float_format_t f)
@cindex @code{least_positive_float ()}
Returns the least positive floating point number (i.e. > 0 but closest to 0)
in float format @code{f}.

@item cl_F least_negative_float (float_format_t f)
@cindex @code{least_negative_float ()}
Returns the least negative floating point number (i.e. < 0 but closest to 0)
in float format @code{f}.

@item cl_F float_epsilon (float_format_t f)
@cindex @code{float_epsilon ()}
Returns the smallest floating point number e > 0 such that @code{1+e != 1}.

@item cl_F float_negative_epsilon (float_format_t f)
@cindex @code{float_negative_epsilon ()}
Returns the smallest floating point number e > 0 such that @code{1-e != 1}.
@end table


@subsection Conversion to rational numbers

Each of the classes @code{cl_R}, @code{cl_RA}, @code{cl_F}
defines the following operation:

@table @code
@item cl_RA rational (const @var{type}& x)
@cindex @code{rational ()}
Returns the value of @code{x} as an exact number. If @code{x} is already
an exact number, this is @code{x}. If @code{x} is a floating-point number,
the value is a rational number whose denominator is a power of 2.
@end table

In order to convert back, say, @code{(cl_F)(cl_R)"1/3"} to @code{1/3}, there is
the function

@table @code
@item cl_RA rationalize (const cl_R& x)
@cindex @code{rationalize ()}
If @code{x} is a floating-point number, it actually represents an interval
of real numbers, and this function returns the rational number with
smallest denominator (and smallest numerator, in magnitude)
which lies in this interval.
If @code{x} is already an exact number, this function returns @code{x}.
@end table

If @code{x} is any float, one has

@itemize @asis
@item
@code{cl_float(rational(x),x) = x}
@item
@code{cl_float(rationalize(x),x) = x}
@end itemize


@section Random number generators


A random generator is a machine which produces (pseudo-)random numbers.
The include file @code{<cln/random.h>} defines a class @code{random_state}
which contains the state of a random generator. If you make a copy
of the random number generator, the original one and the copy will produce
the same sequence of random numbers.

The following functions return (pseudo-)random numbers in different formats.
Calling one of these modifies the state of the random number generator in
a complicated but deterministic way.

The global variable
@cindex @code{random_state}
@cindex @code{default_random_state}
@example
random_state default_random_state
@end example
contains a default random number generator. It is used when the functions
below are called without @code{random_state} argument.

@table @code
@item uint32 random32 (random_state& randomstate)
@itemx uint32 random32 ()
@cindex @code{random32 ()}
Returns a random unsigned 32-bit number. All bits are equally random.

@item cl_I random_I (random_state& randomstate, const cl_I& n)
@itemx cl_I random_I (const cl_I& n)
@cindex @code{random_I ()}
@code{n} must be an integer > 0. This function returns a random integer @code{x}
in the range @code{0 <= x < n}.

@item cl_F random_F (random_state& randomstate, const cl_F& n)
@itemx cl_F random_F (const cl_F& n)
@cindex @code{random_F ()}
@code{n} must be a float > 0. This function returns a random floating-point
number of the same format as @code{n} in the range @code{0 <= x < n}.

@item cl_R random_R (random_state& randomstate, const cl_R& n)
@itemx cl_R random_R (const cl_R& n)
@cindex @code{random_R ()}
Behaves like @code{random_I} if @code{n} is an integer and like @code{random_F}
if @code{n} is a float.
@end table


@section Obfuscating operators
@cindex modifying operators

The modifying C/C++ operators @code{+=}, @code{-=}, @code{*=}, @code{/=},
@code{&=}, @code{|=}, @code{^=}, @code{<<=}, @code{>>=}
are not available by default because their
use tends to make programs unreadable. It is trivial to get away without
them. However, if you feel that you absolutely need these operators
to get happy, then add
@example
#define WANT_OBFUSCATING_OPERATORS
@end example
@cindex @code{WANT_OBFUSCATING_OPERATORS}
to the beginning of your source files, before the inclusion of any CLN
include files. This flag will enable the following operators:

For the classes @code{cl_N}, @code{cl_R}, @code{cl_RA},
@code{cl_F}, @code{cl_SF}, @code{cl_FF}, @code{cl_DF}, @code{cl_LF}:

@table @code
@item @var{type}& operator += (@var{type}&, const @var{type}&)
@cindex @code{operator += ()}
@itemx @var{type}& operator -= (@var{type}&, const @var{type}&)
@cindex @code{operator -= ()}
@itemx @var{type}& operator *= (@var{type}&, const @var{type}&)
@cindex @code{operator *= ()}
@itemx @var{type}& operator /= (@var{type}&, const @var{type}&)
@cindex @code{operator /= ()}
@end table

For the class @code{cl_I}:

@table @code
@item @var{type}& operator += (@var{type}&, const @var{type}&)
@itemx @var{type}& operator -= (@var{type}&, const @var{type}&)
@itemx @var{type}& operator *= (@var{type}&, const @var{type}&)
@itemx @var{type}& operator &= (@var{type}&, const @var{type}&)
@cindex @code{operator &= ()}
@itemx @var{type}& operator |= (@var{type}&, const @var{type}&)
@cindex @code{operator |= ()}
@itemx @var{type}& operator ^= (@var{type}&, const @var{type}&)
@cindex @code{operator ^= ()}
@itemx @var{type}& operator <<= (@var{type}&, const @var{type}&)
@cindex @code{operator <<= ()}
@itemx @var{type}& operator >>= (@var{type}&, const @var{type}&)
@cindex @code{operator >>= ()}
@end table

For the classes @code{cl_N}, @code{cl_R}, @code{cl_RA}, @code{cl_I},
@code{cl_F}, @code{cl_SF}, @code{cl_FF}, @code{cl_DF}, @code{cl_LF}:

@table @code
@item @var{type}& operator ++ (@var{type}& x)
@cindex @code{operator ++ ()}
The prefix operator @code{++x}.

@item void operator ++ (@var{type}& x, int)
The postfix operator @code{x++}.

@item @var{type}& operator -- (@var{type}& x)
@cindex @code{operator -- ()}
The prefix operator @code{--x}.

@item void operator -- (@var{type}& x, int)
The postfix operator @code{x--}.
@end table

Note that by using these obfuscating operators, you wouldn't gain efficiency:
In CLN @samp{x += y;} is exactly the same as  @samp{x = x+y;}, not more
efficient.


@chapter Input/Output
@cindex Input/Output

@section Internal and printed representation
@cindex representation

All computations deal with the internal representations of the numbers.

Every number has an external representation as a sequence of ASCII characters.
Several external representations may denote the same number, for example,
"20.0" and "20.000".

Converting an internal to an external representation is called ``printing'',
@cindex printing
converting an external to an internal representation is called ``reading''.
@cindex reading
In CLN, it is always true that conversion of an internal to an external
representation and then back to an internal representation will yield the
same internal representation. Symbolically: @code{read(print(x)) == x}.
This is called ``print-read consistency''. 

Different types of numbers have different external representations (case
is insignificant):

@table @asis
@item Integers
External representation: @var{sign}@{@var{digit}@}+. The reader also accepts the
Common Lisp syntaxes @var{sign}@{@var{digit}@}+@code{.} with a trailing dot
for decimal integers
and the @code{#@var{n}R}, @code{#b}, @code{#o}, @code{#x} prefixes.

@item Rational numbers
External representation: @var{sign}@{@var{digit}@}+@code{/}@{@var{digit}@}+.
The @code{#@var{n}R}, @code{#b}, @code{#o}, @code{#x} prefixes are allowed
here as well.

@item Floating-point numbers
External representation: @var{sign}@{@var{digit}@}*@var{exponent} or
@var{sign}@{@var{digit}@}*@code{.}@{@var{digit}@}*@var{exponent} or
@var{sign}@{@var{digit}@}*@code{.}@{@var{digit}@}+. A precision specifier
of the form _@var{prec} may be appended. There must be at least
one digit in the non-exponent part. The exponent has the syntax
@var{expmarker} @var{expsign} @{@var{digit}@}+.
The exponent marker is

@itemize @asis
@item
@samp{s} for short-floats,
@item
@samp{f} for single-floats,
@item
@samp{d} for double-floats,
@item
@samp{L} for long-floats,
@end itemize

or @samp{e}, which denotes a default float format. The precision specifying
suffix has the syntax _@var{prec} where @var{prec} denotes the number of
valid mantissa digits (in decimal, excluding leading zeroes), cf. also
function @samp{float_format}.

@item Complex numbers
External representation:
@itemize @asis
@item
In algebraic notation: @code{@var{realpart}+@var{imagpart}i}. Of course,
if @var{imagpart} is negative, its printed representation begins with
a @samp{-}, and the @samp{+} between @var{realpart} and @var{imagpart}
may be omitted. Note that this notation cannot be used when the @var{imagpart}
is rational and the rational number's base is >18, because the @samp{i}
is then read as a digit.
@item
In Common Lisp notation: @code{#C(@var{realpart} @var{imagpart})}.
@end itemize
@end table


@section Input functions

Including @code{<cln/io.h>} defines a number of simple input functions
that read from @code{std::istream&}:

@table @code
@item int freadchar (std::istream& stream)
Reads a character from @code{stream}. Returns @code{cl_EOF} (not a @samp{char}!)
if the end of stream was encountered or an error occurred.

@item int funreadchar (std::istream& stream, int c)
Puts back @code{c} onto @code{stream}. @code{c} must be the result of the
last @code{freadchar} operation on @code{stream}.
@end table

Each of the classes @code{cl_N}, @code{cl_R}, @code{cl_RA}, @code{cl_I},
@code{cl_F}, @code{cl_SF}, @code{cl_FF}, @code{cl_DF}, @code{cl_LF}
defines, in @code{<cln/@var{type}_io.h>}, the following input function:

@table @code
@item std::istream& operator>> (std::istream& stream, @var{type}& result)
Reads a number from @code{stream} and stores it in the @code{result}.
@end table

The most flexible input functions, defined in @code{<cln/@var{type}_io.h>},
are the following:

@table @code
@item cl_N read_complex (std::istream& stream, const cl_read_flags& flags)
@itemx cl_R read_real (std::istream& stream, const cl_read_flags& flags)
@itemx cl_F read_float (std::istream& stream, const cl_read_flags& flags)
@itemx cl_RA read_rational (std::istream& stream, const cl_read_flags& flags)
@itemx cl_I read_integer (std::istream& stream, const cl_read_flags& flags)
Reads a number from @code{stream}. The @code{flags} are parameters which
affect the input syntax. Whitespace before the number is silently skipped.

@item cl_N read_complex (const cl_read_flags& flags, const char * string, const char * string_limit, const char * * end_of_parse)
@itemx cl_R read_real (const cl_read_flags& flags, const char * string, const char * string_limit, const char * * end_of_parse)
@itemx cl_F read_float (const cl_read_flags& flags, const char * string, const char * string_limit, const char * * end_of_parse)
@itemx cl_RA read_rational (const cl_read_flags& flags, const char * string, const char * string_limit, const char * * end_of_parse)
@itemx cl_I read_integer (const cl_read_flags& flags, const char * string, const char * string_limit, const char * * end_of_parse)
Reads a number from a string in memory. The @code{flags} are parameters which
affect the input syntax. The string starts at @code{string} and ends at
@code{string_limit} (exclusive limit). @code{string_limit} may also be
@code{NULL}, denoting the entire string, i.e. equivalent to
@code{string_limit = string + strlen(string)}. If @code{end_of_parse} is
@code{NULL}, the string in memory must contain exactly one number and nothing
more, else a fatal error will be signalled. If @code{end_of_parse}
is not @code{NULL}, @code{*end_of_parse} will be assigned a pointer past
the last parsed character (i.e. @code{string_limit} if nothing came after
the number). Whitespace is not allowed.
@end table

The structure @code{cl_read_flags} contains the following fields:

@table @code
@item cl_read_syntax_t syntax
The possible results of the read operation. Possible values are
@code{syntax_number}, @code{syntax_real}, @code{syntax_rational},
@code{syntax_integer}, @code{syntax_float}, @code{syntax_sfloat},
@code{syntax_ffloat}, @code{syntax_dfloat}, @code{syntax_lfloat}.

@item cl_read_lsyntax_t lsyntax
Specifies the language-dependent syntax variant for the read operation.
Possible values are

@table @code
@item lsyntax_standard
accept standard algebraic notation only, no complex numbers,
@item lsyntax_algebraic
accept the algebraic notation @code{@var{x}+@var{y}i} for complex numbers,
@item lsyntax_commonlisp
accept the @code{#b}, @code{#o}, @code{#x} syntaxes for binary, octal,
hexadecimal numbers,
@code{#@var{base}R} for rational numbers in a given base,
@code{#c(@var{realpart} @var{imagpart})} for complex numbers,
@item lsyntax_all
accept all of these extensions.
@end table

@item unsigned int rational_base
The base in which rational numbers are read.

@item float_format_t float_flags.default_float_format
The float format used when reading floats with exponent marker @samp{e}.

@item float_format_t float_flags.default_lfloat_format
The float format used when reading floats with exponent marker @samp{l}.

@item cl_boolean float_flags.mantissa_dependent_float_format
When this flag is true, floats specified with more digits than corresponding
to the exponent marker they contain, but without @var{_nnn} suffix, will get a
precision corresponding to their number of significant digits.
@end table


@section Output functions

Including @code{<cln/io.h>} defines a number of simple output functions
that write to @code{std::ostream&}:

@table @code
@item void fprintchar (std::ostream& stream, char c)
Prints the character @code{x} literally on the @code{stream}.

@item void fprint (std::ostream& stream, const char * string)
Prints the @code{string} literally on the @code{stream}.

@item void fprintdecimal (std::ostream& stream, int x)
@itemx void fprintdecimal (std::ostream& stream, const cl_I& x)
Prints the integer @code{x} in decimal on the @code{stream}.

@item void fprintbinary (std::ostream& stream, const cl_I& x)
Prints the integer @code{x} in binary (base 2, without prefix)
on the @code{stream}.

@item void fprintoctal (std::ostream& stream, const cl_I& x)
Prints the integer @code{x} in octal (base 8, without prefix)
on the @code{stream}.

@item void fprinthexadecimal (std::ostream& stream, const cl_I& x)
Prints the integer @code{x} in hexadecimal (base 16, without prefix)
on the @code{stream}.
@end table

Each of the classes @code{cl_N}, @code{cl_R}, @code{cl_RA}, @code{cl_I},
@code{cl_F}, @code{cl_SF}, @code{cl_FF}, @code{cl_DF}, @code{cl_LF}
defines, in @code{<cln/@var{type}_io.h>}, the following output functions:

@table @code
@item void fprint (std::ostream& stream, const @var{type}& x)
@itemx std::ostream& operator<< (std::ostream& stream, const @var{type}& x)
Prints the number @code{x} on the @code{stream}. The output may depend
on the global printer settings in the variable @code{default_print_flags}.
The @code{ostream} flags and settings (flags, width and locale) are
ignored.
@end table

The most flexible output function, defined in @code{<cln/@var{type}_io.h>},
are the following:
@example
void print_complex  (std::ostream& stream, const cl_print_flags& flags,
                     const cl_N& z);
void print_real     (std::ostream& stream, const cl_print_flags& flags,
                     const cl_R& z);
void print_float    (std::ostream& stream, const cl_print_flags& flags,
                     const cl_F& z);
void print_rational (std::ostream& stream, const cl_print_flags& flags,
                     const cl_RA& z);
void print_integer  (std::ostream& stream, const cl_print_flags& flags,
                     const cl_I& z);
@end example
Prints the number @code{x} on the @code{stream}. The @code{flags} are
parameters which affect the output.

The structure type @code{cl_print_flags} contains the following fields:

@table @code
@item unsigned int rational_base
The base in which rational numbers are printed. Default is @code{10}.

@item cl_boolean rational_readably
If this flag is true, rational numbers are printed with radix specifiers in
Common Lisp syntax (@code{#@var{n}R} or @code{#b} or @code{#o} or @code{#x}
prefixes, trailing dot). Default is false.

@item cl_boolean float_readably
If this flag is true, type specific exponent markers have precedence over 'E'.
Default is false.

@item float_format_t default_float_format
Floating point numbers of this format will be printed using the 'E' exponent
marker. Default is @code{float_format_ffloat}.

@item cl_boolean complex_readably
If this flag is true, complex numbers will be printed using the Common Lisp
syntax @code{#C(@var{realpart} @var{imagpart})}. Default is false.

@item cl_string univpoly_varname
Univariate polynomials with no explicit indeterminate name will be printed
using this variable name. Default is @code{"x"}.
@end table

The global variable @code{default_print_flags} contains the default values,
used by the function @code{fprint}.


@chapter Rings

CLN has a class of abstract rings.

@example
                         Ring
                       cl_ring
                     <cln/ring.h>
@end example

Rings can be compared for equality:

@table @code
@item bool operator== (const cl_ring&, const cl_ring&)
@itemx bool operator!= (const cl_ring&, const cl_ring&)
These compare two rings for equality.
@end table

Given a ring @code{R}, the following members can be used.

@table @code
@item void R->fprint (std::ostream& stream, const cl_ring_element& x)
@cindex @code{fprint ()}
@itemx cl_boolean R->equal (const cl_ring_element& x, const cl_ring_element& y)
@cindex @code{equal ()}
@itemx cl_ring_element R->zero ()
@cindex @code{zero ()}
@itemx cl_boolean R->zerop (const cl_ring_element& x)
@cindex @code{zerop ()}
@itemx cl_ring_element R->plus (const cl_ring_element& x, const cl_ring_element& y)
@cindex @code{plus ()}
@itemx cl_ring_element R->minus (const cl_ring_element& x, const cl_ring_element& y)
@cindex @code{minus ()}
@itemx cl_ring_element R->uminus (const cl_ring_element& x)
@cindex @code{uminus ()}
@itemx cl_ring_element R->one ()
@cindex @code{one ()}
@itemx cl_ring_element R->canonhom (const cl_I& x)
@cindex @code{canonhom ()}
@itemx cl_ring_element R->mul (const cl_ring_element& x, const cl_ring_element& y)
@cindex @code{mul ()}
@itemx cl_ring_element R->square (const cl_ring_element& x)
@cindex @code{square ()}
@itemx cl_ring_element R->expt_pos (const cl_ring_element& x, const cl_I& y)
@cindex @code{expt_pos ()}
@end table

The following rings are built-in.

@table @code
@item cl_null_ring cl_0_ring
The null ring, containing only zero.

@item cl_complex_ring cl_C_ring
The ring of complex numbers. This corresponds to the type @code{cl_N}.

@item cl_real_ring cl_R_ring
The ring of real numbers. This corresponds to the type @code{cl_R}.

@item cl_rational_ring cl_RA_ring
The ring of rational numbers. This corresponds to the type @code{cl_RA}.

@item cl_integer_ring cl_I_ring
The ring of integers. This corresponds to the type @code{cl_I}.
@end table

Type tests can be performed for any of @code{cl_C_ring}, @code{cl_R_ring},
@code{cl_RA_ring}, @code{cl_I_ring}:

@table @code
@item cl_boolean instanceof (const cl_number& x, const cl_number_ring& R)
@cindex @code{instanceof ()}
Tests whether the given number is an element of the number ring R.
@end table


@chapter Modular integers
@cindex modular integer

@section Modular integer rings
@cindex ring

CLN implements modular integers, i.e. integers modulo a fixed integer N.
The modulus is explicitly part of every modular integer. CLN doesn't
allow you to (accidentally) mix elements of different modular rings,
e.g. @code{(3 mod 4) + (2 mod 5)} will result in a runtime error.
(Ideally one would imagine a generic data type @code{cl_MI(N)}, but C++
doesn't have generic types. So one has to live with runtime checks.)

The class of modular integer rings is

@example
                         Ring
                       cl_ring
                     <cln/ring.h>
                          |
                          |
                 Modular integer ring
                    cl_modint_ring
                  <cln/modinteger.h>
@end example
@cindex @code{cl_modint_ring}

and the class of all modular integers (elements of modular integer rings) is

@example
                    Modular integer
                         cl_MI
                   <cln/modinteger.h>
@end example

Modular integer rings are constructed using the function

@table @code
@item cl_modint_ring find_modint_ring (const cl_I& N)
@cindex @code{find_modint_ring ()}
This function returns the modular ring @samp{Z/NZ}. It takes care
of finding out about special cases of @code{N}, like powers of two
and odd numbers for which Montgomery multiplication will be a win,
@cindex Montgomery multiplication
and precomputes any necessary auxiliary data for computing modulo @code{N}.
There is a cache table of rings, indexed by @code{N} (or, more precisely,
by @code{abs(N)}). This ensures that the precomputation costs are reduced
to a minimum.
@end table

Modular integer rings can be compared for equality:

@table @code
@item bool operator== (const cl_modint_ring&, const cl_modint_ring&)
@cindex @code{operator == ()}
@itemx bool operator!= (const cl_modint_ring&, const cl_modint_ring&)
@cindex @code{operator != ()}
These compare two modular integer rings for equality. Two different calls
to @code{find_modint_ring} with the same argument necessarily return the
same ring because it is memoized in the cache table.
@end table

@section Functions on modular integers

Given a modular integer ring @code{R}, the following members can be used.

@table @code
@item cl_I R->modulus
@cindex @code{modulus}
This is the ring's modulus, normalized to be nonnegative: @code{abs(N)}.

@item cl_MI R->zero()
@cindex @code{zero ()}
This returns @code{0 mod N}.

@item cl_MI R->one()
@cindex @code{one ()}
This returns @code{1 mod N}.

@item cl_MI R->canonhom (const cl_I& x)
@cindex @code{canonhom ()}
This returns @code{x mod N}.

@item cl_I R->retract (const cl_MI& x)
@cindex @code{retract ()}
This is a partial inverse function to @code{R->canonhom}. It returns the
standard representative (@code{>=0}, @code{<N}) of @code{x}.

@item cl_MI R->random(random_state& randomstate)
@itemx cl_MI R->random()
@cindex @code{random ()}
This returns a random integer modulo @code{N}.
@end table

The following operations are defined on modular integers.

@table @code
@item cl_modint_ring x.ring ()
@cindex @code{ring ()}
Returns the ring to which the modular integer @code{x} belongs.

@item cl_MI operator+ (const cl_MI&, const cl_MI&)
@cindex @code{operator + ()}
Returns the sum of two modular integers. One of the arguments may also
be a plain integer.

@item cl_MI operator- (const cl_MI&, const cl_MI&)
@cindex @code{operator - ()}
Returns the difference of two modular integers. One of the arguments may also
be a plain integer.

@item cl_MI operator- (const cl_MI&)
Returns the negative of a modular integer.

@item cl_MI operator* (const cl_MI&, const cl_MI&)
@cindex @code{operator * ()}
Returns the product of two modular integers. One of the arguments may also
be a plain integer.

@item cl_MI square (const cl_MI&)
@cindex @code{square ()}
Returns the square of a modular integer.

@item cl_MI recip (const cl_MI& x)
@cindex @code{recip ()}
Returns the reciprocal @code{x^-1} of a modular integer @code{x}. @code{x}
must be coprime to the modulus, otherwise an error message is issued.

@item cl_MI div (const cl_MI& x, const cl_MI& y)
@cindex @code{div ()}
Returns the quotient @code{x*y^-1} of two modular integers @code{x}, @code{y}.
@code{y} must be coprime to the modulus, otherwise an error message is issued.

@item cl_MI expt_pos (const cl_MI& x, const cl_I& y)
@cindex @code{expt_pos ()}
@code{y} must be > 0. Returns @code{x^y}.

@item cl_MI expt (const cl_MI& x, const cl_I& y)
@cindex @code{expt ()}
Returns @code{x^y}. If @code{y} is negative, @code{x} must be coprime to the
modulus, else an error message is issued.

@item cl_MI operator<< (const cl_MI& x, const cl_I& y)
@cindex @code{operator << ()}
Returns @code{x*2^y}.

@item cl_MI operator>> (const cl_MI& x, const cl_I& y)
@cindex @code{operator >> ()}
Returns @code{x*2^-y}. When @code{y} is positive, the modulus must be odd,
or an error message is issued.

@item bool operator== (const cl_MI&, const cl_MI&)
@cindex @code{operator == ()}
@itemx bool operator!= (const cl_MI&, const cl_MI&)
@cindex @code{operator != ()}
Compares two modular integers, belonging to the same modular integer ring,
for equality.

@item cl_boolean zerop (const cl_MI& x)
@cindex @code{zerop ()}
Returns true if @code{x} is @code{0 mod N}.
@end table

The following output functions are defined (see also the chapter on
input/output).

@table @code
@item void fprint (std::ostream& stream, const cl_MI& x)
@cindex @code{fprint ()}
@itemx std::ostream& operator<< (std::ostream& stream, const cl_MI& x)
@cindex @code{operator << ()}
Prints the modular integer @code{x} on the @code{stream}. The output may depend
on the global printer settings in the variable @code{default_print_flags}.
@end table


@chapter Symbolic data types
@cindex symbolic type

CLN implements two symbolic (non-numeric) data types: strings and symbols.

@section Strings
@cindex string
@cindex @code{cl_string}

The class

@example
                      String
                     cl_string
                   <cln/string.h>
@end example

implements immutable strings.

Strings are constructed through the following constructors:

@table @code
@item cl_string (const char * s)
Returns an immutable copy of the (zero-terminated) C string @code{s}.

@item cl_string (const char * ptr, unsigned long len)
Returns an immutable copy of the @code{len} characters at
@code{ptr[0]}, @dots{}, @code{ptr[len-1]}. NUL characters are allowed.
@end table

The following functions are available on strings:

@table @code
@item operator =
Assignment from @code{cl_string} and @code{const char *}.

@item s.length()
@cindex @code{length ()}
@itemx strlen(s)
@cindex @code{strlen ()}
Returns the length of the string @code{s}.

@item s[i]
@cindex @code{operator [] ()}
Returns the @code{i}th character of the string @code{s}.
@code{i} must be in the range @code{0 <= i < s.length()}.

@item bool equal (const cl_string& s1, const cl_string& s2)
@cindex @code{equal ()}
Compares two strings for equality. One of the arguments may also be a
plain @code{const char *}.
@end table

@section Symbols
@cindex symbol
@cindex @code{cl_symbol}

Symbols are uniquified strings: all symbols with the same name are shared.
This means that comparison of two symbols is fast (effectively just a pointer
comparison), whereas comparison of two strings must in the worst case walk
both strings until their end.
Symbols are used, for example, as tags for properties, as names of variables
in polynomial rings, etc.

Symbols are constructed through the following constructor:

@table @code
@item cl_symbol (const cl_string& s)
Looks up or creates a new symbol with a given name.
@end table

The following operations are available on symbols:

@table @code
@item cl_string (const cl_symbol& sym)
Conversion to @code{cl_string}: Returns the string which names the symbol
@code{sym}.

@item bool equal (const cl_symbol& sym1, const cl_symbol& sym2)
@cindex @code{equal ()}
Compares two symbols for equality. This is very fast.
@end table


@chapter Univariate polynomials
@cindex polynomial
@cindex univariate polynomial

@section Univariate polynomial rings

CLN implements univariate polynomials (polynomials in one variable) over an
arbitrary ring. The indeterminate variable may be either unnamed (and will be
printed according to @code{default_print_flags.univpoly_varname}, which
defaults to @samp{x}) or carry a given name. The base ring and the
indeterminate are explicitly part of every polynomial. CLN doesn't allow you to
(accidentally) mix elements of different polynomial rings, e.g.
@code{(a^2+1) * (b^3-1)} will result in a runtime error. (Ideally this should
return a multivariate polynomial, but they are not yet implemented in CLN.)

The classes of univariate polynomial rings are

@example
                           Ring
                         cl_ring
                       <cln/ring.h>
                            |
                            |
                 Univariate polynomial ring
                      cl_univpoly_ring
                      <cln/univpoly.h>
                            |
           +----------------+-------------------+
           |                |                   |
 Complex polynomial ring    |    Modular integer polynomial ring
 cl_univpoly_complex_ring   |        cl_univpoly_modint_ring
 <cln/univpoly_complex.h>   |        <cln/univpoly_modint.h>
                            |
           +----------------+
           |                |
   Real polynomial ring     |
   cl_univpoly_real_ring    |
   <cln/univpoly_real.h>    |
                            |
           +----------------+
           |                |
 Rational polynomial ring   |
 cl_univpoly_rational_ring  |
 <cln/univpoly_rational.h>  |
                            |
           +----------------+
           |
 Integer polynomial ring
 cl_univpoly_integer_ring
 <cln/univpoly_integer.h>
@end example

and the corresponding classes of univariate polynomials are

@example
                   Univariate polynomial
                          cl_UP
                      <cln/univpoly.h>
                            |
           +----------------+-------------------+
           |                |                   |
   Complex polynomial       |      Modular integer polynomial
        cl_UP_N             |                cl_UP_MI
 <cln/univpoly_complex.h>   |        <cln/univpoly_modint.h>
                            |
           +----------------+
           |                |
     Real polynomial        |
        cl_UP_R             |
  <cln/univpoly_real.h>     |
                            |
           +----------------+
           |                |
   Rational polynomial      |
        cl_UP_RA            |
 <cln/univpoly_rational.h>  |
                            |
           +----------------+
           |
   Integer polynomial
        cl_UP_I
 <cln/univpoly_integer.h>
@end example

Univariate polynomial rings are constructed using the functions

@table @code
@item cl_univpoly_ring find_univpoly_ring (const cl_ring& R)
@itemx cl_univpoly_ring find_univpoly_ring (const cl_ring& R, const cl_symbol& varname)
This function returns the polynomial ring @samp{R[X]}, unnamed or named.
@code{R} may be an arbitrary ring. This function takes care of finding out
about special cases of @code{R}, such as the rings of complex numbers,
real numbers, rational numbers, integers, or modular integer rings.
There is a cache table of rings, indexed by @code{R} and @code{varname}.
This ensures that two calls of this function with the same arguments will
return the same polynomial ring.

@itemx cl_univpoly_complex_ring find_univpoly_ring (const cl_complex_ring& R)
@cindex @code{find_univpoly_ring ()}
@itemx cl_univpoly_complex_ring find_univpoly_ring (const cl_complex_ring& R, const cl_symbol& varname)
@itemx cl_univpoly_real_ring find_univpoly_ring (const cl_real_ring& R)
@itemx cl_univpoly_real_ring find_univpoly_ring (const cl_real_ring& R, const cl_symbol& varname)
@itemx cl_univpoly_rational_ring find_univpoly_ring (const cl_rational_ring& R)
@itemx cl_univpoly_rational_ring find_univpoly_ring (const cl_rational_ring& R, const cl_symbol& varname)
@itemx cl_univpoly_integer_ring find_univpoly_ring (const cl_integer_ring& R)
@itemx cl_univpoly_integer_ring find_univpoly_ring (const cl_integer_ring& R, const cl_symbol& varname)
@itemx cl_univpoly_modint_ring find_univpoly_ring (const cl_modint_ring& R)
@itemx cl_univpoly_modint_ring find_univpoly_ring (const cl_modint_ring& R, const cl_symbol& varname)
These functions are equivalent to the general @code{find_univpoly_ring},
only the return type is more specific, according to the base ring's type.
@end table

@section Functions on univariate polynomials

Given a univariate polynomial ring @code{R}, the following members can be used.

@table @code
@item cl_ring R->basering()
@cindex @code{basering ()}
This returns the base ring, as passed to @samp{find_univpoly_ring}.

@item cl_UP R->zero()
@cindex @code{zero ()}
This returns @code{0 in R}, a polynomial of degree -1.

@item cl_UP R->one()
@cindex @code{one ()}
This returns @code{1 in R}, a polynomial of degree == 0.

@item cl_UP R->canonhom (const cl_I& x)
@cindex @code{canonhom ()}
This returns @code{x in R}, a polynomial of degree <= 0.

@item cl_UP R->monomial (const cl_ring_element& x, uintL e)
@cindex @code{monomial ()}
This returns a sparse polynomial: @code{x * X^e}, where @code{X} is the
indeterminate.

@item cl_UP R->create (sintL degree)
@cindex @code{create ()}
Creates a new polynomial with a given degree. The zero polynomial has degree
@code{-1}. After creating the polynomial, you should put in the coefficients,
using the @code{set_coeff} member function, and then call the @code{finalize}
member function.
@end table

The following are the only destructive operations on univariate polynomials.

@table @code
@item void set_coeff (cl_UP& x, uintL index, const cl_ring_element& y)
@cindex @code{set_coeff ()}
This changes the coefficient of @code{X^index} in @code{x} to be @code{y}.
After changing a polynomial and before applying any "normal" operation on it,
you should call its @code{finalize} member function.

@item void finalize (cl_UP& x)
@cindex @code{finalize ()}
This function marks the endpoint of destructive modifications of a polynomial.
It normalizes the internal representation so that subsequent computations have
less overhead. Doing normal computations on unnormalized polynomials may
produce wrong results or crash the program.
@end table

The following operations are defined on univariate polynomials.

@table @code
@item cl_univpoly_ring x.ring ()
@cindex @code{ring ()}
Returns the ring to which the univariate polynomial @code{x} belongs.

@item cl_UP operator+ (const cl_UP&, const cl_UP&)
@cindex @code{operator + ()}
Returns the sum of two univariate polynomials.

@item cl_UP operator- (const cl_UP&, const cl_UP&)
@cindex @code{operator - ()}
Returns the difference of two univariate polynomials.

@item cl_UP operator- (const cl_UP&)
Returns the negative of a univariate polynomial.

@item cl_UP operator* (const cl_UP&, const cl_UP&)
@cindex @code{operator * ()}
Returns the product of two univariate polynomials. One of the arguments may
also be a plain integer or an element of the base ring.

@item cl_UP square (const cl_UP&)
@cindex @code{square ()}
Returns the square of a univariate polynomial.

@item cl_UP expt_pos (const cl_UP& x, const cl_I& y)
@cindex @code{expt_pos ()}
@code{y} must be > 0. Returns @code{x^y}.

@item bool operator== (const cl_UP&, const cl_UP&)
@cindex @code{operator == ()}
@itemx bool operator!= (const cl_UP&, const cl_UP&)
@cindex @code{operator != ()}
Compares two univariate polynomials, belonging to the same univariate
polynomial ring, for equality.

@item cl_boolean zerop (const cl_UP& x)
@cindex @code{zerop ()}
Returns true if @code{x} is @code{0 in R}.

@item sintL degree (const cl_UP& x)
@cindex @code{degree ()}
Returns the degree of the polynomial. The zero polynomial has degree @code{-1}.

@item sintL ldegree (const cl_UP& x)
@cindex @code{degree ()}
Returns the low degree of the polynomial. This is the degree of the first
non-vanishing polynomial coefficient. The zero polynomial has ldegree @code{-1}.

@item cl_ring_element coeff (const cl_UP& x, uintL index)
@cindex @code{coeff ()}
Returns the coefficient of @code{X^index} in the polynomial @code{x}.

@item cl_ring_element x (const cl_ring_element& y)
@cindex @code{operator () ()}
Evaluation: If @code{x} is a polynomial and @code{y} belongs to the base ring,
then @samp{x(y)} returns the value of the substitution of @code{y} into
@code{x}.

@item cl_UP deriv (const cl_UP& x)
@cindex @code{deriv ()}
Returns the derivative of the polynomial @code{x} with respect to the
indeterminate @code{X}.
@end table

The following output functions are defined (see also the chapter on
input/output).

@table @code
@item void fprint (std::ostream& stream, const cl_UP& x)
@cindex @code{fprint ()}
@itemx std::ostream& operator<< (std::ostream& stream, const cl_UP& x)
@cindex @code{operator << ()}
Prints the univariate polynomial @code{x} on the @code{stream}. The output may
depend on the global printer settings in the variable
@code{default_print_flags}.
@end table

@section Special polynomials

The following functions return special polynomials.

@table @code
@item cl_UP_I tschebychev (sintL n)
@cindex @code{tschebychev ()}
@cindex Chebyshev polynomial
Returns the n-th Chebyshev polynomial (n >= 0).

@item cl_UP_I hermite (sintL n)
@cindex @code{hermite ()}
@cindex Hermite polynomial
Returns the n-th Hermite polynomial (n >= 0).

@item cl_UP_RA legendre (sintL n)
@cindex @code{legendre ()}
@cindex Legende polynomial
Returns the n-th Legendre polynomial (n >= 0).

@item cl_UP_I laguerre (sintL n)
@cindex @code{laguerre ()}
@cindex Laguerre polynomial
Returns the n-th Laguerre polynomial (n >= 0).
@end table

Information how to derive the differential equation satisfied by each
of these polynomials from their definition can be found in the
@code{doc/polynomial/} directory.


@chapter Internals

@section Why C++ ?
@cindex advocacy

Using C++ as an implementation language provides

@itemize @bullet
@item
Efficiency: It compiles to machine code.

@item
@cindex portability
Portability: It runs on all platforms supporting a C++ compiler. Because
of the availability of GNU C++, this includes all currently used 32-bit and
64-bit platforms, independently of the quality of the vendor's C++ compiler.

@item
Type safety: The C++ compilers knows about the number types and complains if,
for example, you try to assign a float to an integer variable. However,
a drawback is that C++ doesn't know about generic types, hence a restriction
like that @code{operator+ (const cl_MI&, const cl_MI&)} requires that both
arguments belong to the same modular ring cannot be expressed as a compile-time
information.

@item
Algebraic syntax: The elementary operations @code{+}, @code{-}, @code{*},
@code{=}, @code{==}, ... can be used in infix notation, which is more
convenient than Lisp notation @samp{(+ x y)} or C notation @samp{add(x,y,&z)}.
@end itemize

With these language features, there is no need for two separate languages,
one for the implementation of the library and one in which the library's users
can program. This means that a prototype implementation of an algorithm
can be integrated into the library immediately after it has been tested and
debugged. No need to rewrite it in a low-level language after having prototyped
in a high-level language.


@section Memory efficiency

In order to save memory allocations, CLN implements:

@itemize @bullet
@item
Object sharing: An operation like @code{x+0} returns @code{x} without copying
it.
@item
@cindex garbage collection
@cindex reference counting
Garbage collection: A reference counting mechanism makes sure that any
number object's storage is freed immediately when the last reference to the
object is gone.
@item
@cindex immediate numbers
Small integers are represented as immediate values instead of pointers
to heap allocated storage. This means that integers @code{> -2^29},
@code{< 2^29} don't consume heap memory, unless they were explicitly allocated
on the heap.
@end itemize


@section Speed efficiency

Speed efficiency is obtained by the combination of the following tricks
and algorithms:

@itemize @bullet
@item
Small integers, being represented as immediate values, don't require
memory access, just a couple of instructions for each elementary operation.
@item
The kernel of CLN has been written in assembly language for some CPUs
(@code{i386}, @code{m68k}, @code{sparc}, @code{mips}, @code{arm}).
@item
On all CPUs, CLN may be configured to use the superefficient low-level
routines from GNU GMP version 3.
@item
For large numbers, CLN uses, instead of the standard @code{O(N^2)}
algorithm, the Karatsuba multiplication, which is an
@iftex
@tex
$O(N^{1.6})$
@end tex
@end iftex
@ifinfo
@code{O(N^1.6)}
@end ifinfo
algorithm.
@item
For very large numbers (more than 12000 decimal digits), CLN uses
@iftex
Sch{@"o}nhage-Strassen
@cindex Sch{@"o}nhage-Strassen multiplication
@end iftex
@ifinfo
Schnhage-Strassen
@cindex Schnhage-Strassen multiplication
@end ifinfo
multiplication, which is an asymptotically optimal multiplication 
algorithm.
@item
These fast multiplication algorithms also give improvements in the speed
of division and radix conversion.
@end itemize


@section Garbage collection
@cindex garbage collection

All the number classes are reference count classes: They only contain a pointer
to an object in the heap. Upon construction, assignment and destruction of
number objects, only the objects' reference count are manipulated.

Memory occupied by number objects are automatically reclaimed as soon as
their reference count drops to zero.

For number rings, another strategy is implemented: There is a cache of,
for example, the modular integer rings. A modular integer ring is destroyed
only if its reference count dropped to zero and the cache is about to be
resized. The effect of this strategy is that recently used rings remain
cached, whereas undue memory consumption through cached rings is avoided.


@chapter Using the library

For the following discussion, we will assume that you have installed
the CLN source in @code{$CLN_DIR} and built it in @code{$CLN_TARGETDIR}.
For example, for me it's @code{CLN_DIR="$HOME/cln"} and
@code{CLN_TARGETDIR="$HOME/cln/linuxelf"}. You might define these as
environment variables, or directly substitute the appropriate values.


@section Compiler options
@cindex compiler options

Until you have installed CLN in a public place, the following options are
needed:

When you compile CLN application code, add the flags
@example
   -I$CLN_DIR/include -I$CLN_TARGETDIR/include
@end example
to the C++ compiler's command line (@code{make} variable CFLAGS or CXXFLAGS).
When you link CLN application code to form an executable, add the flags
@example
   $CLN_TARGETDIR/src/libcln.a
@end example
to the C/C++ compiler's command line (@code{make} variable LIBS).

If you did a @code{make install}, the include files are installed in a
public directory (normally @code{/usr/local/include}), hence you don't
need special flags for compiling. The library has been installed to a
public directory as well (normally @code{/usr/local/lib}), hence when
linking a CLN application it is sufficient to give the flag @code{-lcln}.

Since CLN version 1.1, there are two tools to make the creation of
software packages that use CLN easier:
@itemize @bullet
@item
@cindex @code{cln-config}
@code{cln-config} is a shell script that you can use to determine the
compiler and linker command line options required to compile and link a
program with CLN.  Start it with @code{--help} to learn about its options
or consult the manpage that comes with it.
@item
@cindex @code{AC_PATH_CLN}
@code{AC_PATH_CLN} is for packages configured using GNU automake.
The synopsis is:
@example
@code{AC_PATH_CLN([@var{MIN-VERSION}, [@var{ACTION-IF-FOUND} [, @var{ACTION-IF-NOT-FOUND}]]])}
@end example
This macro determines the location of CLN using @code{cln-config}, which
is either found in the user's path, or from the environment variable
@code{CLN_CONFIG}.  It tests the installed libraries to make sure that
their version is not earlier than @var{MIN-VERSION} (a default version
will be used if not specified). If the required version was found, sets
the @env{CLN_CPPFLAGS} and the @env{CLN_LIBS} variables. This
macro is in the file @file{cln.m4} which is installed in
@file{$datadir/aclocal}. Note that if automake was installed with a
different @samp{--prefix} than CLN, you will either have to manually
move @file{cln.m4} to automake's @file{$datadir/aclocal}, or give
aclocal the @samp{-I} option when running it. Here is a possible example
to be included in your package's @file{configure.ac}:
@example
AC_PATH_CLN(1.1.0, [
  LIBS="$LIBS $CLN_LIBS"
  CPPFLAGS="$CPPFLAGS $CLN_CPPFLAGS"
], AC_MSG_ERROR([No suitable installed version of CLN could be found.]))
@end example
@end itemize


@section Compatibility to old CLN versions
@cindex namespace
@cindex compatibility

As of CLN version 1.1 all non-macro identifiers were hidden in namespace
@code{cln} in order to avoid potential name clashes with other C++
libraries. If you have an old application, you will have to manually
port it to the new scheme. The following principles will help during
the transition:
@itemize @bullet
@item
All headers are now in a separate subdirectory. Instead of including
@code{cl_}@var{something}@code{.h}, include
@code{cln/}@var{something}@code{.h} now.
@item
All public identifiers (typenames and functions) have lost their
@code{cl_} prefix.  Exceptions are all the typenames of number types,
(cl_N, cl_I, cl_MI, @dots{}), rings, symbolic types (cl_string,
cl_symbol) and polynomials (cl_UP_@var{type}).  (This is because their
names would not be mnemonic enough once the namespace @code{cln} is
imported. Even in a namespace we favor @code{cl_N} over @code{N}.)
@item
All public @emph{functions} that had by a @code{cl_} in their name still
carry that @code{cl_} if it is intrinsic part of a typename (as in
@code{cl_I_to_int ()}).
@end itemize
When developing other libraries, please keep in mind not to import the
namespace @code{cln} in one of your public header files by saying
@code{using namespace cln;}. This would propagate to other applications
and can cause name clashes there.


@section Include files
@cindex include files
@cindex header files

Here is a summary of the include files and their contents.

@table @code
@item <cln/object.h>
General definitions, reference counting, garbage collection.
@item <cln/number.h>
The class cl_number.
@item <cln/complex.h>
Functions for class cl_N, the complex numbers.
@item <cln/real.h>
Functions for class cl_R, the real numbers.
@item <cln/float.h>
Functions for class cl_F, the floats.
@item <cln/sfloat.h>
Functions for class cl_SF, the short-floats.
@item <cln/ffloat.h>
Functions for class cl_FF, the single-floats.
@item <cln/dfloat.h>
Functions for class cl_DF, the double-floats.
@item <cln/lfloat.h>
Functions for class cl_LF, the long-floats.
@item <cln/rational.h>
Functions for class cl_RA, the rational numbers.
@item <cln/integer.h>
Functions for class cl_I, the integers.
@item <cln/io.h>
Input/Output.
@item <cln/complex_io.h>
Input/Output for class cl_N, the complex numbers.
@item <cln/real_io.h>
Input/Output for class cl_R, the real numbers.
@item <cln/float_io.h>
Input/Output for class cl_F, the floats.
@item <cln/sfloat_io.h>
Input/Output for class cl_SF, the short-floats.
@item <cln/ffloat_io.h>
Input/Output for class cl_FF, the single-floats.
@item <cln/dfloat_io.h>
Input/Output for class cl_DF, the double-floats.
@item <cln/lfloat_io.h>
Input/Output for class cl_LF, the long-floats.
@item <cln/rational_io.h>
Input/Output for class cl_RA, the rational numbers.
@item <cln/integer_io.h>
Input/Output for class cl_I, the integers.
@item <cln/input.h>
Flags for customizing input operations.
@item <cln/output.h>
Flags for customizing output operations.
@item <cln/malloc.h>
@code{malloc_hook}, @code{free_hook}.
@item <cln/abort.h>
@code{cl_abort}.
@item <cln/condition.h>
Conditions/exceptions.
@item <cln/string.h>
Strings.
@item <cln/symbol.h>
Symbols.
@item <cln/proplist.h>
Property lists.
@item <cln/ring.h>
General rings.
@item <cln/null_ring.h>
The null ring.
@item <cln/complex_ring.h>
The ring of complex numbers.
@item <cln/real_ring.h>
The ring of real numbers.
@item <cln/rational_ring.h>
The ring of rational numbers.
@item <cln/integer_ring.h>
The ring of integers.
@item <cln/numtheory.h>
Number threory functions.
@item <cln/modinteger.h>
Modular integers.
@item <cln/V.h>
Vectors.
@item <cln/GV.h>
General vectors.
@item <cln/GV_number.h>
General vectors over cl_number.
@item <cln/GV_complex.h>
General vectors over cl_N.
@item <cln/GV_real.h>
General vectors over cl_R.
@item <cln/GV_rational.h>
General vectors over cl_RA.
@item <cln/GV_integer.h>
General vectors over cl_I.
@item <cln/GV_modinteger.h>
General vectors of modular integers.
@item <cln/SV.h>
Simple vectors.
@item <cln/SV_number.h>
Simple vectors over cl_number.
@item <cln/SV_complex.h>
Simple vectors over cl_N.
@item <cln/SV_real.h>
Simple vectors over cl_R.
@item <cln/SV_rational.h>
Simple vectors over cl_RA.
@item <cln/SV_integer.h>
Simple vectors over cl_I.
@item <cln/SV_ringelt.h>
Simple vectors of general ring elements.
@item <cln/univpoly.h>
Univariate polynomials.
@item <cln/univpoly_integer.h>
Univariate polynomials over the integers.
@item <cln/univpoly_rational.h>
Univariate polynomials over the rational numbers.
@item <cln/univpoly_real.h>
Univariate polynomials over the real numbers.
@item <cln/univpoly_complex.h>
Univariate polynomials over the complex numbers.
@item <cln/univpoly_modint.h>
Univariate polynomials over modular integer rings.
@item <cln/timing.h>
Timing facilities.
@item <cln/cln.h>
Includes all of the above.
@end table


@section An Example

A function which computes the nth Fibonacci number can be written as follows.
@cindex Fibonacci number

@example
#include <cln/integer.h>
#include <cln/real.h>
using namespace cln;

// Returns F_n, computed as the nearest integer to
// ((1+sqrt(5))/2)^n/sqrt(5). Assume n>=0.
const cl_I fibonacci (int n)
@{
        // Need a precision of ((1+sqrt(5))/2)^-n.
        float_format_t prec = float_format((int)(0.208987641*n+5));
        cl_R sqrt5 = sqrt(cl_float(5,prec));
        cl_R phi = (1+sqrt5)/2;
        return round1( expt(phi,n)/sqrt5 );
@}
@end example

Let's explain what is going on in detail.

The include file @code{<cln/integer.h>} is necessary because the type
@code{cl_I} is used in the function, and the include file @code{<cln/real.h>}
is needed for the type @code{cl_R} and the floating point number functions.
The order of the include files does not matter.  In order not to write
out @code{cln::}@var{foo} in this simple example we can safely import
the whole namespace @code{cln}.

Then comes the function declaration. The argument is an @code{int}, the
result an integer. The return type is defined as @samp{const cl_I}, not
simply @samp{cl_I}, because that allows the compiler to detect typos like
@samp{fibonacci(n) = 100}. It would be possible to declare the return
type as @code{const cl_R} (real number) or even @code{const cl_N} (complex
number). We use the most specialized possible return type because functions
which call @samp{fibonacci} will be able to profit from the compiler's type
analysis: Adding two integers is slightly more efficient than adding the
same objects declared as complex numbers, because it needs less type
dispatch. Also, when linking to CLN as a non-shared library, this minimizes
the size of the resulting executable program.

The result will be computed as expt(phi,n)/sqrt(5), rounded to the nearest
integer. In order to get a correct result, the absolute error should be less
than 1/2, i.e. the relative error should be less than sqrt(5)/(2*expt(phi,n)).
To this end, the first line computes a floating point precision for sqrt(5)
and phi.

Then sqrt(5) is computed by first converting the integer 5 to a floating point
number and than taking the square root. The converse, first taking the square
root of 5, and then converting to the desired precision, would not work in
CLN: The square root would be computed to a default precision (normally
single-float precision), and the following conversion could not help about
the lacking accuracy. This is because CLN is not a symbolic computer algebra
system and does not represent sqrt(5) in a non-numeric way.

The type @code{cl_R} for sqrt5 and, in the following line, phi is the only
possible choice. You cannot write @code{cl_F} because the C++ compiler can
only infer that @code{cl_float(5,prec)} is a real number. You cannot write
@code{cl_N} because a @samp{round1} does not exist for general complex
numbers.

When the function returns, all the local variables in the function are
automatically reclaimed (garbage collected). Only the result survives and
gets passed to the caller.

The file @code{fibonacci.cc} in the subdirectory @code{examples}
contains this implementation together with an even faster algorithm.

@section Debugging support
@cindex debugging

When debugging a CLN application with GNU @code{gdb}, two facilities are
available from the library:

@itemize @bullet
@item The library does type checks, range checks, consistency checks at
many places. When one of these fails, the function @code{cl_abort()} is
called. Its default implementation is to perform an @code{exit(1)}, so
you won't have a core dump. But for debugging, it is best to set a
breakpoint at this function:
@example
(gdb) break cl_abort
@end example
When this breakpoint is hit, look at the stack's backtrace:
@example
(gdb) where
@end example

@item The debugger's normal @code{print} command doesn't know about
CLN's types and therefore prints mostly useless hexadecimal addresses.
CLN offers a function @code{cl_print}, callable from the debugger,
for printing number objects. In order to get this function, you have
to define the macro @samp{CL_DEBUG} and then include all the header files
for which you want @code{cl_print} debugging support. For example:
@cindex @code{CL_DEBUG}
@example
#define CL_DEBUG
#include <cln/string.h>
@end example
Now, if you have in your program a variable @code{cl_string s}, and
inspect it under @code{gdb}, the output may look like this:
@example
(gdb) print s
$7 = @{<cl_gcpointer> = @{ = @{pointer = 0x8055b60, heappointer = 0x8055b60,
  word = 134568800@}@}, @}
(gdb) call cl_print(s)
(cl_string) ""
$8 = 134568800
@end example
Note that the output of @code{cl_print} goes to the program's error output,
not to gdb's standard output.

Note, however, that the above facility does not work with all CLN types,
only with number objects and similar. Therefore CLN offers a member function
@code{debug_print()} on all CLN types. The same macro @samp{CL_DEBUG}
is needed for this member function to be implemented. Under @code{gdb},
you call it like this:
@cindex @code{debug_print ()}
@example
(gdb) print s
$7 = @{<cl_gcpointer> = @{ = @{pointer = 0x8055b60, heappointer = 0x8055b60,
  word = 134568800@}@}, @}
(gdb) call s.debug_print()
(cl_string) ""
(gdb) define cprint
>call ($1).debug_print()
>end
(gdb) cprint s
(cl_string) ""
@end example
Unfortunately, this feature does not seem to work under all circumstances.
@end itemize


@chapter Customizing
@cindex customizing

@section Error handling

When a fatal error occurs, an error message is output to the standard error
output stream, and the function @code{cl_abort} is called. The default
version of this function (provided in the library) terminates the application.
To catch such a fatal error, you need to define the function @code{cl_abort}
yourself, with the prototype
@example
#include <cln/abort.h>
void cl_abort (void);
@end example
@cindex @code{cl_abort ()}
This function must not return control to its caller.


@section Floating-point underflow
@cindex underflow

Floating point underflow denotes the situation when a floating-point number
is to be created which is so close to @code{0} that its exponent is too
low to be represented internally. By default, this causes a fatal error.
If you set the global variable
@example
cl_boolean cl_inhibit_floating_point_underflow
@end example
to @code{cl_true}, the error will be inhibited, and a floating-point zero
will be generated instead.  The default value of 
@code{cl_inhibit_floating_point_underflow} is @code{cl_false}.


@section Customizing I/O

The output of the function @code{fprint} may be customized by changing the
value of the global variable @code{default_print_flags}.
@cindex @code{default_print_flags}


@section Customizing the memory allocator

Every memory allocation of CLN is done through the function pointer
@code{malloc_hook}. Freeing of this memory is done through the function
pointer @code{free_hook}. The default versions of these functions,
provided in the library, call @code{malloc} and @code{free} and check
the @code{malloc} result against @code{NULL}.
If you want to provide another memory allocator, you need to define
the variables @code{malloc_hook} and @code{free_hook} yourself,
like this:
@example
#include <cln/malloc.h>
namespace cln @{
        void* (*malloc_hook) (size_t size) = @dots{};
        void (*free_hook) (void* ptr)      = @dots{};
@}
@end example
@cindex @code{malloc_hook ()}
@cindex @code{free_hook ()}
The @code{cl_malloc_hook} function must not return a @code{NULL} pointer.

It is not possible to change the memory allocator at runtime, because
it is already called at program startup by the constructors of some
global variables.




@c Indices

@unnumbered Index

@printindex my


@bye
